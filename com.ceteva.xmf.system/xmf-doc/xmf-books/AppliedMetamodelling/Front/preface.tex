\addcontentsline{toc}{chapter}{Preface (Second Edition)}
\chapter*{Preface}
\LARGE{\emph{(Second Edition)}}\\[2cm]
\normalsize

Since publishing the first edition in 2004 we were pleasantly surprised by how much momentum
\emph{Applied Metamodelling} created.  The fundamental tenets of the book have been cited in a vast
number of papers and books, and key ideas have been influential in a number of technology
initiatives including the Eclipse Modeling Project \cite{EMP}.

A recurring response that we
commonly receive is a frustration that although the book gives readers the knowledge to apply
metamodelling, they cannot easily do this using the XMF technology which was used to construct
the examples in the book.  This is about to change with the release of XMF as an open source
technology in the first quarter of 2008.

Much has happened since 2004 but the fundamental problem of complexity in software and systems
remains.  This second version of the book makes a number of corrections to the original text and
includes a number of new chapters including two additional case studies which demonstrate the
application and benefits of metamodelling.

\addcontentsline{toc}{chapter}{Preface (First Edition)}
\chapter*{Preface}
\LARGE{\emph{(First Edition)}}\\[2cm]
\normalsize

Modern day system developers have some serious problems to contend
with. The systems they develop are becoming increasingly complex
as customers demand richer functionality delivered in ever shorter
timescales. They have to manage a huge diversity of implementation
technologies, design techniques and development processes:
everything from scripting languages to web-services to the latest
'silver bullet' design abstraction. To add to that, nothing stays
still: today's 'must have' technology rapidly becomes tomorrow's
legacy problem that must be managed along with everything else.

How can these problems be dealt with? In this book we propose that
there is a common foundation to their resolution: {\em languages}.
Languages are the primary way in which system developers
communicate, design and implement systems. Languages provide
abstractions that can encapsulate complexity, embrace the
diversity of technologies and design abstractions, and unite
modern and legacy systems.

\section*{Language-Driven Development}

Understanding how we can manage languages to best fit the needs of
developers is the key to improving system development practises.
We call this {\em Language-Driven Development}. The right
languages enable developers to be significantly more productive
than using traditional development practices. Rather than dealing
with a plethora of low level technologies, developers can use
powerful language abstractions and development environments that
support their development processes. They can create models that
are rich enough to permit analysis and simulation of system
properties before completely generating the code for the system.
They can manipulate their models and programs in significantly
more sophisticated ways than they can code. Moreover, provided the
language definitions are flexible, they can adapt their languages
to meet their development needs with relative ease.

\section*{Metamodelling}

In order to realise Language-Driven Development, we need the
ability to rapidly design and integrate semantically rich
languages in a unified way. {\em Metamodelling} is the way to
achieve this. A metamodel is a model of a language that captures
its essential properties and features. These include the language
concepts it supports, its textual and/or graphical syntax and its
semantics (what the models and programs written in the language
mean and how they behave). Metamodels unify languages because the
same metamodelling language is used in each case. Significant
advantage can be made of this unification to construct powerful
and flexible design environments for languages. These enable the
rapid assembly of Language-Driven Development tools that give
developers the power they need to design their systems faster,
cheaper and more flexibly.

\section*{Purpose of this Book}

The aim of this book is to advance the state of the art in
metamodelling to the point at which it can realise the
Language-Driven Development vision. Traditionally, metamodelling
has focused on the design of data centric models of language
concepts. In this book, we go way beyond that, showing how they
can capture all aspects of languages, including concrete syntax,
abstract syntax and semantics in a sound and pragmatic way.

This book also aims to fill a large gap in the metamodelling
literature, providing a technically rich book on an a subject that
is mentioned often, but for which there is little concrete
material available. Metamodels are increasingly being used across
wider application domains, and it is the intention that this book
will provide good advice to metamodellers irrespective of the
languages and tools they are using.

\section*{Scope of this Book}

The scope of this book is deliberately not restricted to software
systems. Many other types of domains, from systems engineering to
business, manufacturing and even physical engineering, can benefit
from the ideas presented here.

\section*{Intended Audience}

This book should be useful to anyone who has an interest in the
design of languages, language and model-driven development, and
metamodelling. Our intention is that the book has an industrial
focus, and we have tried hard to ensure that it is relevant to
real practitioners. In particular, everything in this book has
been implemented, therefore ensuring that it is has been tried and
tested.

\section*{Relationship to Other Approaches}

We do not claim that the ideas presented in this book are new.
Approaches to rapidly designing languages in flexible ways have
been around since the early days of LISP, ObjVLisp and Smalltalk.
Meta-case tools have been widely touted as a means of designing
tailored tools and languages. Efforts by the Object Management
Group (OMG) to standardise facilities for capturing language
meta-data are already influencing the way vendors build tools.
More recently, work on domain specific languages has highlighted
the benefits of rapidly designing languages targeted at specific
application domains.

In this book we have combined many of these approaches on top of
existing standards to facilitate the definition of languages in a
general and complete way. An important emphasis has been on
raising the level of abstraction at which complex aspects of
language design such as the definition of concrete syntax and
semantics are expressed. Thus we {\em model} languages, but in
sufficient detail that these models can be turned into
semantically rich development environments and tools. This
capability has not been achieved in such a complete way before.

\section*{Organisation of this Book}

This book is organised into three parts. The first and shortest
part gives an overview of challenges facing the system development
industry and proposes Language-Driven Development as a way forward
to addressing those challenges.

The middle part provides a detailed treatment of metamodelling. It
contains the following chapters:

\begin{description}
\item[Metamodelling]: introduces the key features of languages and
describes what metamodels are and how they can capture those
features. \item[A Metamodelling Facility]: presents an overview of
an executable metamodelling facility that provides a number of
powerful languages for capturing the syntax and semantics of
languages. \item[Abstract Syntax]: describes how metamodels can be
used to define the concepts that are provided by the language.
\item[Concrete Syntax]: describes how the textual and
diagrammatical syntaxes of language can be modelled. \item [
Semantics]: introduces semantics and the motivation for having
them in language definitions. The chapter goes on to describe four
different approach to describing semantics. \item[Executable
Metamodelling]: discusses how the addition of a small number of
action primitives to a metamodelling language turn it into a
powerful metaprogramming environment for Language-Driven
Development. \item[Mappings]: motivates and presents two languages
for transforming and relating metamodels. \item[Reuse]: this
chapter describes a number of different approaches to reusing
existing language metamodels.
\end{description}

The final part provides a number of indepth case studies each describing a
specific example of metamodelling.  These examples range from the design of
a small general purpose language to a domain specific language for modelling
interactive television based applications.  The case studies are a significant
resource for demonstrating metamodelling best practice.

\section*{Acknowledgements}

We are grateful to the following people for giving us feedback on
early versions of the book and the technology it is based on:
Manooch Azmoodeh, Steve Crook-Dawkings, Tom Dalton, John Davies,
Kevin Dockerill, Desmond D'Souza, Karl Frank, Nektarios Georgal,
Robert France, Sheila Gupta, Martin Hargreaves, Girish Maskeri,
Stephen Mellor, Joaquin Miller, Alan Moore, Alphonse Orosun, John
Rowlands, Bran Selic, Laurence Tratt, Andrew Watson, Stephen
Whymark.
