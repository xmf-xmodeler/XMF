\chapter{Case Study 3: A Simple Programming Language}

\section{Introduction}
\label{Intro} This chapter describes how XMF can be used to define
the syntax and semantics of a simple action based programming
language called XAction. It begins by presenting the XAction
language and then define a concrete grammar for it. Next a variety
of approaches, including operational and translational approaches
are used to define an executable semantics for the language. Note,
this chapter provides an in depth technical treatment of
semantics.

\section{XAction}

{\em XAction} is a simple action language with values that are
either records or are atomic. An atomic data value is a string,
integer or boolean. A record is a collection of named values.
XAction is block-structured where blocks contain type definitions
and value definitions. XAction has simple control structures:
conditional statements and loops. The following is a simple
example XAction program that builds a list of even numbers from 2
to 100:

\small
\begin{verbatim}
begin
  type Pair is head tail end
  type Nil is end
  value length is 100 end
  value list is new Nil end
  while length > 0 do
    begin
      if length % 2 = 0
      then
        begin
          value pair is new Pair end
          pair.head := length;
          pair.tail := list;
          list := pair;
        end
      end;
      length := length - 1;
    end
end
\end{verbatim}
\normalsize

The definition of XAction is structured as a collection of XMF
packages. The {\tt Values} package defines the semantic domain for
XAction; it contains classes for each type of program value.
Executable program phrases in XAction are divided into two
categories: {\tt Expressions} and {\tt Statements}. Expressions
evaluate to produce XAction values. Statements are used to control
the flow of execution and to update values.
\begin{lstlisting}
@Package XAction
  @Package Values end
  @Package Expressions end
  @Package Statements end
end
\end{lstlisting}
The rest of this section defines the syntax of XAction by giving
the basic class definitions and the XBNF grammar rules for the
language constructs.

\subsection{XAction Values}

XAction expressions evaluate to produce XAction values. Values are
defined in the {\tt Values} package and which is the {\em semantic
domain} for XAction. Values are either atomic: integers and
booleans, or are records. We use a simple representation for
records: a sequence of values indexed by names.

XAction records are created by instantiating XAction record types.
A record type is a sequence of names. Types raise an interesting
design issue: should the types be included as part of the semantic
domain since evaluation of certain XAction program phrases give
rise to types that are used later in the execution to produce
records. The answer to the question involves the phase distinction
that occurs between {\em static} analysis (or execution) and {\em
dynamic} execution. Types are often viewed as occurring only
during static analysis; although this is not always the case. We
will show how the semantics of XAction can be defined with and
without dynamic types.

All values are instances of sub-classes of the class {\tt Value}:
\begin{lstlisting}
context Values
 @Class Value
   @Attribute value : Element end
   @Constructor(value) ! end
 end
\end{lstlisting}
Atomic values are either booleans or integers. Each class defines
operations that the semantic domain provides for manipulating
XAction values. The classes below show the structure and a
representative sample of operations:
\begin{lstlisting}
context Values
  @Class Bool extends Value
    @Operation binAnd(Bool(b))
      Bool(value and b)
    end
    @Operation binOr(Bool(b))
      Bool(value or b)
    end
  end
\end{lstlisting}\begin{lstlisting}
context Values
  @Class Int extends Value
    @Operation binAdd(Int(n))
      Int(value + n)
    end
  end
\end{lstlisting}
Record types are sequences of names. A type provides a {\tt new}
operation that instantiates the type to produce a new record. This
operation is only meaningful if we have dynamic types:
\begin{lstlisting}
context Values
  @Class Type extends Value
    @Attribute names : Seq(String) end
    @Constructor(names) ! end
    @Operation new()
      Record(self,names->collect(n | Seq{n | null}))
    end
  end
\end{lstlisting}
Records are sequences of values indexed by names; the names are
found by navigating to the type of the record:
\begin{lstlisting}
context Values
  @Class Record extends Value
    @Attribute type : Type end
    @Attribute fields : Seq(Element) end
    @Constructor(type,fields) ! end
    @Operation lookup(name:String)
      fields->at(type.names->indexOf(name))
    end
    @Operation update(name:String,value:Element)
      fields->setAt(type.names->indexOf(name),value)
    end
  end
\end{lstlisting}\subsection{XAction Expressions}

XAction expressions are program phrases that evaluate to produce
XAction values. The following classes define the expression types:
\begin{lstlisting}
context Expressions
  @Class Exp
  end
end
\end{lstlisting}
A binary expression has a left and right sub-expression and an
operation. The name of the operation is represented as a string:
\begin{lstlisting}
context Expressions
  @Class BinExp extends Exp
    @Attribute op : String end
    @Attribute left : Exp end
    @Attribute right : Exp end
    @Constructor(op,left,right) ! end
  end
\end{lstlisting}
An atomic constant expression is either an integer or a boolean:
\begin{lstlisting}
context Expressions
  @Class Const extends Exp
    @Attribute value : Element end
    @Constructor(value) ! end
  end
end
\end{lstlisting}
A new record is produced by performing a {\tt new} expression. The
type to instantiate is given as a string. An alternative
representation for types in {\tt new} expressions would be to
permit an arbitrary expression that {\em evaluates} to produce a
type. This design choice would rule out static typing and force
the language to have dynamic types. We wish to use XAction to
illustrate the difference between dynamic and static types in
semantic definitions so we use strings to name types in {\tt new}
expressions:
\begin{lstlisting}
context Expressions
  @Class New extends Exp
    @Attribute type : String end
    @Constructor(type) ! end
  end
end
\end{lstlisting}
A variable is just a name:
\begin{lstlisting}
context Expressions
  @Class Var extends Exp
    @Attribute name : String end
    @Constructor(name) ! end
  end
\end{lstlisting}
A record field ref is:
\begin{lstlisting}
context Expressions
  @Class FieldRef extends Exp
    @Attribute value : Exp end
    @Attribute name : String end
    @Constructor(value,name) ! end
  end
\end{lstlisting}
The concrete syntax of expressions is defined by the XBNF grammar
for the class {\tt Exp}. The grammar parses the expression syntax
and synthesizes instances of the expression classes:
\begin{lstlisting}
context Exp
 @Grammar
   // Start at Exp. Logical operators bind weakest.
   Exp ::= e = ArithExp [ op = LogicalOp l = Exp { BinExp(op,e,l) } ].
   LogicalOp ::= 'and' { "and" } | 'or' { "or" }.
   // The '.' for field ref binds tighter than '+' etc.
   ArithExp ::= e = FieldRef [ op = ArithOp a = FieldRef { BinExp(op,e,a) } ].
   ArithOp ::= '+' { "+" }.
   // A field reference '.' optionally follows an atomic expression.
   FieldRef ::= e = Atom ('.' n = Name { FieldRef(e,n) } | { e }).
   // Atomic expressions can be arbitrary exps if in ( and ).
   Atom ::= Const | Var | New | '(' Exp ')'.
   Const ::= IntConst | BoolConst.
   IntConst ::= i = Int { Const(i) }.
   BoolConst ::= 'true' { Const(true) } | 'false' { Const(false) }.
   Var ::= n = Name { Var(n) }.
   New ::= 'new' n = Name { New(n) }.
 end
\end{lstlisting}\subsection{XAction Statements}

XAction statements are used to:
\begin{itemize}
\item Introduce new names associated with either types or values.
\item Control the flow of execution. \item Perform side effects on
records.
\end{itemize}
The following classes define the statement types for XAction:
\begin{lstlisting}
context Statements
  @Class Statement
  end
end
\end{lstlisting}
A block (as in Pascal or C) contains local definitions. Names
introduced in a block are available for the rest of the statements
in the block (including sub-blocks) but are not available when
control exits from the block:
\begin{lstlisting}
context Statements
  @Class Block extends Statement
    @Attribute statements : Seq(Statement) end
    @Constructor(statements) ! end
  end
end
\end{lstlisting}
A declaration introduces either a type or a value binding:
\begin{lstlisting}
context Statements
  @Class Declaration isabstract extends Statement
    @Attribute name : String end
  end
end
\end{lstlisting}
A type declaration associates a type name with a sequence of field
names. To keep things simple we don't associate fields with types:
\begin{lstlisting}
context Statements
  @Class TypeDeclaration extends Declaration
    @Attribute names : Seq(String) end
    @Constructor(name,names) ! end
  end
end
\end{lstlisting}
A value declaration associates a name with a new value. The value
is produced by performing an expression at run-time:
\begin{lstlisting}
context Statements
  @Class ValueDeclaration extends Declaration
    @Attribute value : Exp end
    @Constructor(name,value) ! end
  end
end
\end{lstlisting}
A while statement involves a test and a body:
\begin{lstlisting}
context Statements
  @Class While extends Declaration
    @Attribute test : Exp end
    @Attribute body : Statement end
    @Constructor(test,body) ! end
  end
end
\end{lstlisting}
An if statement involves a test, a then-part and an else-part:
\begin{lstlisting}
context Statements
  @Class If extends Declaration
    @Attribute test : Exp end
    @Attribute thenPart : Statement end
    @Attribute elsePart : Statement end
    @Constructor(test,elsePart) ! end
  end
end
\end{lstlisting}\begin{lstlisting}
context Statements
  @Class FieldUpdate extends Declaration
    @Attribute record : Exp end
    @Attribute name : Exp end
    @Attribute value : Exp end
    @Constructor(record,name,value) ! end
  end
end
\end{lstlisting}\begin{lstlisting}
context Statements
  @Class Update extends Declaration
    @Attribute name : String end
    @Attribute value : Exp end
    @Constructor(name,value) ! end
  end
end
\end{lstlisting}\begin{lstlisting}
context Statement
  @Grammar extends Exp.grammar
    Statement ::= Block | Declaration | While | If | Update | FieldUpdate.
    Block ::= 'begin' s = Statement* 'end' { Block(s) }.
    Declaration ::= TypeDeclaration | ValueDeclaration.
    TypeDeclaration ::= 'type' n = Name 'is' ns = Name* 'end' {
      TypeDeclaration(n,ns) }.
    ValueDeclaration ::= 'value' n = Name 'is' e = Exp 'end' {
      ValueDeclaration(n,e) }.
    FieldUpdate ::= e = Exp '.' n = Name ':=' v = Exp ';' {
      FieldUpdate(e,n,v) }.
    While ::= 'while' e = Exp 'do' s = Statement 'end' {
      While(e,s) }.
    If ::= 'if' e = Exp 'then' s1 = Statement 'else' s2 = Statement 'end' {
      If(e,s1,s2) }.
    Update ::= n = Name ':=' e = Exp ';' {
      Update(n,e) }.
  end
\end{lstlisting}\section{An Evaluator for XAction}

As described in the introducion we are interested in defining
XAction operational semantics. We will do this in a number of
different ways in the rest of this note. The first, and possibly
most straightforward, approach is to define an {\em interpreter}
for XAction in the XOCL language. This involves writing an {\tt
eval} operation for each of the XAction syntax classes. The {\tt
eval} operation must be parameterized with respect to any context
information that is required to perform the evaluation. An XAction
program {\tt p} is then evaluated in a context {\tt e} by: {\tt
p.eval(e)}.

\subsection{Evaluating Expressions}
Expression evaluation is defined by adding {\tt eval} operations
to each class in {\tt Expressions} as follows:
\begin{lstlisting}
context Exp
  @AbstractOp eval(env:Env):Value
  end
\end{lstlisting}
Evaluation of a constant produces the appropriate semantic domain
value:
\begin{lstlisting}
context Const
  @Operation eval(env)
    @TypeCase(value)
      Boolean do Bool(value) end
      Integer do Int(value) end
    end
  end
\end{lstlisting}
Evaluation of a variable involves looking up the current value.
The value is found in the current context of evaluation: this must
contain associations between variable names and their values. This
is the only thing required of the XAction evaluation context and
therefore we represent the context as an {\em environment} of
variable bindings:
\begin{lstlisting}
context Var
  @Operation eval(env)
    env.lookup(name)
  end
\end{lstlisting}
Evaluation of a binary expression involves evaluation of the
sub-expressions and then selecting an operation based on the
operation name. The following shows how XAction semantics is
completely based on XOCl semantics since {\tt +} in XAction is
performed by {\tt +} in XOCL.
\begin{lstlisting}
context BinExp
  @Operation eval(env)
    @Case op of
      "and" do left.eval(env).binAnd(right.eval(env)) end
      "or"  do left.eval(env).binOr(right.eval(env)) end
      "+"   do left.eval(env).binAdd(right.eval(env)) end
    end
  end
\end{lstlisting}
Creation of new records is performed by evaluaing a {\tt new}
expression. The interpreter has dynamic types so the type to
instantiate is found by looking up the type name in the current
environment:
\begin{lstlisting}
context New
  @Operation eval(env)
    env.lookup(type).new()
  end
\end{lstlisting}
Field reference is defined as follows:
\begin{lstlisting}
context FieldRef
  @Operation eval(env)
    value.eval(env).lookup(name)
  end
\end{lstlisting}\subsection{Evaluating Statements}

XAction statements are performed in order to introduce new names,
control flow or to update a record field. Statements are defined
to evaluate in a context and must observe the rules of scope that
require variables are local to the block that introduces them. The
context of execution is an environment; evaluation of a statement
may update the supplied environment, so statement evaluation
returns an environment:
\begin{lstlisting}
context Statement
  @AbstractOp eval(env):Env
  end
\end{lstlisting}
A value declaration evaluates the expression part and then extends
the supplied environment with a new binding:
\begin{lstlisting}
context ValueDeclaration
  @Operation eval(env)
    env.bind(name,value.eval(env))
  end
\end{lstlisting}
A type declaration extends the supplied environment with a new
type:
\begin{lstlisting}
context TypeDeclaration
  @Operation eval(env)
    env.bind(name,Type(names))
  end
\end{lstlisting}
A block must preserve the supplied environment when its evaluation
is complete. Each statement in the block is performed in turn and
may update the current environment:
\begin{lstlisting}
context Block
  @Operation eval(originalEnv)
    let env = originalEnv
    in @For statement in statements do
         env := statement.eval(env)
       end
    end;
    originalEnv
    end
\end{lstlisting}
A {\tt while} statement continually performs the body while the
test expression returns {\tt true}. A while body is equivalent to
a block; so any updates to the supplied environment that are
performed by the while body are discarded on exit:
\begin{lstlisting}
context While
  @Operation eval(originalEnv)
    let env = orginalEnv
    in @While test.eval(env).value do
          env := body.eval(env)
       end;
       originalEnv
    end
  end
\end{lstlisting}
An {\tt if} statement conditionally performs one of its
sub-statements:
\begin{lstlisting}
context If
  @Operation eval(env)
    if test.eval(env).value
    then thenPart.eval(env)
    else elsePart.eval(env)
    end
  end
\end{lstlisting}\begin{lstlisting}
context FieldUpdate
  @Operation eval(env)
    record.eval(env).update(name,value.eval(env))
  end
\end{lstlisting}\begin{lstlisting}
context Update
  @Operation eval(env)
    env.update(name,value.eval(env))
  end
\end{lstlisting}\section{A Translator for XAction with Run-Time Types}

The previous section defines an interpreter for XAction. This is
an appealing way to define the operational semantics of a language
because the rules of evaluation work directly on the abstract
syntax structures. However the resulting interpreter can often be
very inefficient. Furthermore, an interpreter can lead to an {\em
evaluation phase distinction}. Suppose that XAction is to be
embedded in XOCL. XOCL has its own interpretive mechanism (the XMF
VM); at the boundary between XOCL and XAction the XOCL
interpretive mechanism must hand over to the XAction interpreter
-- the XAction code that is performed is a data structure, a
completely alien format to the VM. This phase distinction can lead
to problems when using standard tools, such as save and load
mechanisms, with respect to the new language. For example a
mechanism that can save XOCL code to disk cannot be used to save
XAction code to disk (it can, however, be used to save the XAction
interpreter to disk).

An alternative strategy is to translate the source code of XAction
to a language for which we have an efficient implementation.  No
new interpretive mechanism is required and no phase distinction
arises. Translation provides the opportunity for static analysis
(since translation is performed prior to executing the program).
As we mentioned earlier, static analysis can translate out any
type information from XAction programs; the resulting program does
not require run-time types. Since static analysis requires a
little more work, this section describes a simple translation from
XAction to XOCL that results in run-time types; the subsequent
section shows how this can be extended to analyse types statically
and remove them from the semantic domain.

\subsection{Translating Expressions}

Translation is defined by adding a new operation {\tt desugar1} to
each sbatract syntax class. There is no static analysis, so the
operation does not require any arguments. The result of the
operation is a value of type {\tt Performable} which is the type
of elements that can be executed by the XMF execution engine.
\begin{lstlisting}
context Exp
  @AbstractOp desugar1():Performable
  end
\end{lstlisting}
An XAction constant is translated to an XOCL constant:
\begin{lstlisting}
context Const
  @Operation desugar1():Performable
    @TypeCase(value)
      Boolean do BoolExp(value) end
      Integer do IntExp(value) end
    end
  end
\end{lstlisting}
An XAction binary expression is translated to an XOCL binary
expression. Note that the sub-expressions are also translated:
\begin{lstlisting}
context BinExp
  @Operation desugar1():Performable
    @Case op of
      "and" do [| <left.desugar1()> and <right.desugar1()> |] end
      "or"  do [| <left.desugar1()> and <right.desugar1()> |] end
      "+"   do [| <left.desugar1()> + <right.desugar1()> |] end
    end
  end
\end{lstlisting}
An XAction {\tt new} expression involves a type name. Types will
be bound to the appropriate variable name in the resulting XOCL
program; so the result of translation is just a message {\tt new}
sent to the value of the variable whose name is the type name:
\begin{lstlisting}
context New
  @Operation desugar1():Performable
    [| <OCL::Var(type)>.new() |]
  end
\end{lstlisting}
XAction variables are translated to XOCL variables:
\begin{lstlisting}
context Var
  @Operation desugar1():Performable
    OCL::Var(name)
  end
\end{lstlisting}
XAction field references are translated to the appropriate call on
a record:
\begin{lstlisting}
context FieldRef
  @Operation desugar1():Performable
    [| <value.desugar1()>.ref(<StrExp(name)>) |]
  end
\end{lstlisting}\subsection{Translating Statements}

An XAction statement can involve local blocks. The equivalent XOCL
expression that provides local definitions is {\tt let}. A {\tt
let} expression consists of a name, a value expression and a body
expression. Thus, in order to translate an XAction declaration to
an XOCL {\tt let} we need to be passed the body of the {\tt let}.
This leads to a translational style for XAction actions called
{\em continuation passing} where each {\tt desugar1} operation is
supplied with the XOCL action that will be performed next:
\begin{lstlisting}
context Statement
  @AbstractOp desugar1(next:Performable):Performable
  end
\end{lstlisting}
A type declaration is translated to a local definition for the
type name. Note that the expression {\tt names.lift()} translates
the sequence of names to an expression that, when performed,
produces the same sequence of names: {\tt list} is a means of
performing evaluation in reverse:
\begin{lstlisting}
context TypeDeclaration
  @Operation desugar1(next:Performable):Performable
    [| let <name> = Type(<names.lift()>)
       in <next>
       end
    |]
  end
\end{lstlisting}
A value declaration is translated to a local decinition:
\begin{lstlisting}
context ValueDeclaration
  @Operation desugar1(next:Performable):Performable
    [| let <name> = <value.desugar1()>
       in <next>
       end
    |]
  end
\end{lstlisting}
A block requires each sub-statement to be translated in turn.
Continuation passing allows us to chain together the sequence of
statements and nest the local definitions appropriately. The
following auxiliary operation is used to implement
block-translation:
\begin{lstlisting}
context Statements
  @Operation desugar1(statements,next:Performable):Performable
    @Case statements of
      Seq{} do
        next
      end
      Seq{statement | statements} do
        statement.desugar1(Statements::desugar1(statements,next))
      end
    end
  end
\end{lstlisting}
Translation of a block requires that the XOCL local definitions
are kept local. Therefore, the sub-statements are translated by
chaining them together and with a final continuation of {\tt
null}. Placing the result in sequence with {\tt next} ensures that
any definitions are local to the block.
\begin{lstlisting}
context Block
  @Operation desugar1(next:Performable):Performable
    [| <Statements::desugar1(statements,[| null |])> ;
       <next>
    |]
  end
\end{lstlisting}
A {\tt while} statement is translated to the equivalent expression
in XOCL:
\begin{lstlisting}
context While
  @Operation desugar1(next:Performable):Performable
    [| @While <test.desugar1()>.value do
         <body.desugar1([|null|])>
       end;
       <next>
    |]
  end
\end{lstlisting}
An {\tt if} statement is translated to an equivalent expression in
XOCL:
\begin{lstlisting}
context If
  @Operation desugar1(next:Performable):Performable
    [| if <test.desugar1()>.value
       then <thenPart.desugar1(next)>
       else <elsePart.desugar1(next)>
       end
    |]
  end
\end{lstlisting}\begin{lstlisting}
context FieldUpdate
  @Operation desugar1(next:Performable):Performable
    [| <record.desugar1()>.update(<StrExp(name)>,<value.desugar1()>);
       <next>
    |]
  end
\end{lstlisting}\begin{lstlisting}
context Update
  @Operation desugar1(next:Performable):Performable
    [| <name> := <value.desugar1()>;
       <next>
    |]
  end
\end{lstlisting}\section{A Translator for XAction without Run-Time Types}

It is usual for languages to have a static (or {\em compile time})
phase and a dynamic (or {\em run time}) phase. Many operational
features of the language can be performed statically. This
includes type analysis: checking that types are defined before
they are used and allocating appropriate structures when instances
of types are created. This section shows how the translator for
XAction to XOCL from the previous section can be modified so that
type analysis is performed and so that types do not occur at
run-time.

\subsection{Translating Expressions}

Since types will no longer occur at run-time we will simplify the
semantic domain slightly and represent records as {\em a-lists}.
An a-list is a sequence of pairs, the first element of each pair
is a ket and the second element is a value. In this case a record
is an a-list where the keys are field name strings. XOCL provides
operations defined on sequences that are to be used as a-lists:
{\tt l->lookup(key)} and {\tt l->set(key,value)}.

The context for static analysis is a type environment. Types now
occur at translation time instead of run-time therefore that
portion of the run-time context that would contain associations
between type names and types occurs during translation:
\begin{lstlisting}
context Exp
  @AbstractOp desugar2(typeEnv:Env):Performable
  end
\end{lstlisting}
Translation of a constant is as for {\tt desugar1}:
\begin{lstlisting}
context Const
  @Operation desugar2(typeEnv:Env):Performable
    self.desugar1()
  end
\end{lstlisting}
Translation of binary expressions is as for {\tt desugar1} except
that all translation is performed by {\tt desugar2}:
\begin{lstlisting}
context BinExp
  @Operation desugar2(typeEnv:Env):Performable
    @Case op of
      "and" do [| <left.desugar2(typeEnv)> and
                  <right.desugar2(typeEnv)> |] end
      "or"  do [| <left.desugar2(typeEnv)> and
                  <right.desugar2(typeEnv)> |] end
      "+"   do [| <left.desugar2(typeEnv)> +
                  <right.desugar2(typeEnv)> |] end
    end
  end
\end{lstlisting}
Translation of a variable is as before:
\begin{lstlisting}
context Var
  @Operation desugar2(typeEnv:Env):Performable
    self.desugar1()
  end
\end{lstlisting}
A {\tt new} expression involves a reference to a type name. The
types occur at translation time and therefore part of the
evaluation of {\tt new} can occur during translation. The type
should occur in the supplied type environment; the type contains
the sequence of field names. The result of translation is an XOCL
expression that constructs an a-list based on the names of the
fields in the type. The initial value for each field is {\tt
null}:
\begin{lstlisting}
context New
  @Operation desugar2(typeEnv:Env):Performable
    if typeEnv.binds(type)
    then
      let type = typeEnv.lookup(type)
      in type.names->iterate(name exp = [| Seq{} |] |
           [| <exp>->bind(<StrExp(name)>,null) |])
      end
    else self.error("Unknown type " + type)
    end
  end
\end{lstlisting}
A field reference expression is translated to an a-list lookup
expression:
\begin{lstlisting}
context FieldRef
  @Operation desugar2(typeEnv:Env):Performable
    [| <value.desugar2(typeEnv)>->lookup(<StrExp(name)>) |]
  end
\end{lstlisting}\subsection{Translating Statements}

A statement may contain a local type definition. We have already
discussed continuation passing with respect to {\tt desugar1}
where the context for translation includes the next XOCL
expression to perform. The {\tt desugar2} operation cannot be
supplied with the next XOCL expression because this will depend on
whether or not the current statement extends the type environment.
Therefore, in {\tt desugar2} the continuation is an operation that
is awaiting a type environment and produces the next XOCL
expression:
\begin{lstlisting}
context Statement
  @AbstractOp desugar2(typeExp:Env,next:Operation):Performable
  end
\end{lstlisting}
A type declaration binds the type at translation time and supplies
the extended type environment to the continuation:
\begin{lstlisting}
context TypeDeclaration
  @Operation desugar2(typeEnv:Env,next:Operation):Performable
    next(typeEnv.bind(name,Type(names)))
  end
\end{lstlisting}
A value declaration introduces a new local definition; the body is
created by supplying the unchanged type environment to the
continuation:
\begin{lstlisting}
context ValueDeclaration
  @Operation desugar2(typeEnv:Env,next:Operation):Performable
    [| let <name> = <value.desugar2(typeEnv)>
       in <next(typeEnv)>
       end
    |]
  end
\end{lstlisting}
Translation of a block involves translation of a sequence of
sub-statements. The following auxiliary operation ensures that the
continuations are chained together correctly:
\begin{lstlisting}
context Statements
  @Operation desugar2(statements,typeEnv,next):Performable
      @Case statements of
        Seq{} do
          next(typeEnv)
        end
        Seq{statement | statements} do
          statement.desugar2(
            typeEnv,
            @Operation(typeEnv)
              Statements::desugar2(statements,typeEnv,next)
            end)
        end
      end
    end
\end{lstlisting}
A block is translated to a sequence of statements where local
definitions are implemented using nested {\tt let} expressions in
XOCL. The locality of the definitions is maintained by sequencing
the block statements and the continuation expression:
\begin{lstlisting}
context Block
  @Operation desugar2(typeEnv:Env,next:Operation):Performable
    [| <Statements::desugar2(
         statements,
         typeEnv,
         @Operation(ignore)
           [| null |]
         end)>;
       <next(typeEnv)>
    |]
  end
\end{lstlisting}
A {\tt while} statement is translated so that the XOCL expression
is in sequence with the expression produced by the contintuation:
\begin{lstlisting}
context While
  @Operation desugar2(typeEnv:Env,next:Operation):Performable
    [| @While <test.desugar2(typeEnv)>.value do
         <body.desugar2(typeEnv,@Operation(typeEnv) [| null |] end)>
         end;
         <next(typeEnv)>
       end
    |]
  end
\end{lstlisting}
The {\tt if} statement is translated to an equivalent XOCL
expression:
\begin{lstlisting}
context If
  @Operation desugar2(typeEnv:Env,next:Operation):Performable
    [| if <test.desugar2(typeEnv)>.value
       then <thenPart.desugar2(typeEnv,next)>
       else <elsePart.desugar2(typeEnv,next)>
       end
    |]
  end
\end{lstlisting}\begin{lstlisting}
context FieldUpdate
  @Operation desugar2(typeEnv:Env,next:Operation):Performable
    [| <record.desugar2(typeEnv)>.update(
          <StrExp(name)>,
          <value.desugar2(typeEnv)>);
       <next(typeEnv)>
    |]
  end
\end{lstlisting}\begin{lstlisting}
context Update
  @Operation desugar2(typeEnv:Env,next:Operation):Performable
    [| <name> := <value.desugar2(typeEnv)>;
       <next(typeEnv)>
    |]
  end
\end{lstlisting}\section{Compiling XAction}

The previous section shows how to perform static type anslysis
while translating XAction to XOCL. XOCL is then translated to XMF
VM instructions by the XOCL compiler (another translation
process). The result is that XAction cannot to anything that XOCL
cannot do. Whilst this is not a serious restriction, there may be
times where a new language wishes to translate directly to the XMF
VM without going through an existing XMF language. This may be in
order to produce highly efficient code, or because the language
has some unusual control constructs that XOCL does not support.
This section shows how XAction can be translated directly to XMF
VM instructions.

\subsection{Compiling Expressions}

\begin{lstlisting}
context Exp
  @AbstractOp compile(typeEnv:Env,valueEnv:Seq(String)):Seq(Instr)
  end
\end{lstlisting}\begin{lstlisting}
context Const
  @Operation compile(typeEnv,valueEnv)
    @TypeCase(value)
      Boolean do
        if value
        then Seq{PushTrue()}
        else Seq{PushFalse()}
        end
      end
      Integer do
        Seq{PushInteger(value)}
      end
    end
  end
\end{lstlisting}\begin{lstlisting}
context Var
  @Operation compile(typeEnv,valueEnv)
    let index = valueEnv->indexOf(name)
    in if index < 0
       then self.error("Unbound variable " + name)
       else Seq{LocalRef(index)}
       end
    end
  end
\end{lstlisting}\begin{lstlisting}
context BinExp
  @Operation compile(typeEnv,valueEnv):Seq(Instr)
    left.compile(typeEnv,valueEnv) +
    right.compile(typeEnv,valueEnv) +
    @Case op of
      "and" do Seq{And()} end
      "or"  do Seq{Or()} end
      "+"   do Seq{Add()} end
    end
  end
\end{lstlisting}\begin{lstlisting}
context New
  @Operation compile(typeEnv,valueEnv):Seq(Instr)
    self.desugar2(typeEnv).compile()
  end
\end{lstlisting}\begin{lstlisting}
context FieldRef
  @Operation compile(typeEnv,valueEnv):Seq(Instr)
    Seq{StartCall(),
        PushStr(name)}
    value.compile(typeExp,valueExp) +
    Seq{Send("lookup",1)}
  end
\end{lstlisting}\subsection{Compiling Statements}

\begin{lstlisting}
context Statement
  @AbstractOp compile(typeEnv:Env,varEnv:Seq(String),next:Operation):Seq(Instr)
  end
\end{lstlisting}\begin{lstlisting}
context TypeDeclaration
  @Operation compile(typeEnv,varEnv,next)
    next(typeEnv.bind(name,Type(names)),varEnv)
  end
\end{lstlisting}\begin{lstlisting}
context ValueDeclaration
  @Operation compile(typeEnv,varEnv,next)
    value.compile(typeEnv,varEnv) +
    Seq{SetLocal(name,varEnv->size),
        Pop()} +
    next(typeEnv,varEnv + Seq{name})
  end
\end{lstlisting}\begin{lstlisting}
context Statements
  @Operation compile(statements,typeEnv,varEnv,next)
    @Case statements of
      Seq{} do
        next(typeEnv,varEnv)
      end
      Seq{statement | statements} do
        statement.compile(
          typeEnv,
          varEnv,
          @Operation(typeEnv,varEnv)
            Statements::compile(statements,typeEnv,varEnv,next)
          end)
      end
    end
  end
\end{lstlisting}\begin{lstlisting}
context Block
  @Operation compile(typeEnv,varEnv,next)
    Statements::compile(
      statements,
      typeEnv,
      varEnv,
      @Operation(localTypeEnv,localVarEnv)
        next(typeEnv,varEnv)
        end)
  end
\end{lstlisting}\begin{lstlisting}
context While
  @Operation compile(typeEnv,varEnv,next)
     Seq{Noop("START")} +
     test.compile(typeEnv,varEnv) +
     Seq{SkipFalse("END")} +
     body.compile(typeEnv,varEnv,
       @Operation(typeEnv,varEnv)
         Seq{}
       end) +
     Seq{Skip("START")} +
     Seq{Noop("END")} +
     next(typeEnv,varEnv)
  end
\end{lstlisting}\begin{lstlisting}
context If
  @Operation compile(typeEnv,varEnv,next)
     test.compile(typeEnv,varEnv) +
     Seq{SkipFalse("ELSE")} +
     thenPart.compile(typeEnv,varEnv,
       @Operation(typeEnv,varEnv)
         Seq{Skip("END")}
       end) +
     Seq{Noop("ELSE")} +
     elsePart.compile(typeEnv,varEnv,
       @Operation(typeEnv,varEnv)
         Seq{Skip("END")}
       end) +
     Seq{Noop("END")} +
     next(typeEnv,varEnv)
  end
\end{lstlisting}\section{Abstract Syntax to Concrete Syntax}

We have shown how XAction is translated from concrete syntax to
abstract syntax by defining an XBNF grammar. It is often useful to
be able to translate in the opposite direction and produce
concrete syntax from abstract syntax. This can be done with or
without formatting. The latter is useful only when the concrete
syntax is to be consumed by a machine or when it can be supplied
to a pretty-printing tool.

Formatting of code can be performed in fairly sophisticated ways,
for example allowing the width of the page to be supplied as a
parameter to the formatter. This section shows how a simple code
formatter for XAction can be defined by attaching {\tt pprint}
operations to the abstract syntax classes.

An expression is formatted by supplying it with an output channel,
it is assumed that the channel is in the correct output column:
\begin{lstlisting}
@AbstractOp pprint(out:OutputChannel) end
\end{lstlisting}
A variable is pretty-printed by printing its name:
\begin{lstlisting}
context Var
  @Operation pprint(out)
    format(out,"~S",Seq{name})
  end
\end{lstlisting}
A constant is pretty-printed by printing its value:
\begin{lstlisting}
context Const
  @Operation pprint(out)
    format(out,"~S",Seq{value})
  end
\end{lstlisting}
A {\tt new} expression prepends the type with the keyword:
\begin{lstlisting}
context New
  @Operation pprint(out)
    format(out,"new ~S",Seq{type})
  end
\end{lstlisting}
A binary expression pretty-prints the left sub-expression, the
operator name and then the right sub-expression:
\begin{lstlisting}
context BinExp
  @Operation pprint(out)
    left.pprint(out);
    format(out," ~S ",Seq{op});
    right.pprint(out)
  end
\end{lstlisting}
A statement is pretty-printed by supplying it with the output
channel and the current level of indentation. The indentation
controls how many tab-stops must be output after each newline.
This is necessary because statements can be nested and indentation
is used to visualise the level of nesting.
\begin{lstlisting}
context Statement
  @AbstractOp pprint(out:OutputChannel,indent:Integer) end
\end{lstlisting}
A block is pretty-printed by incrementing the indentation for each
nested statement:
\begin{lstlisting}
context Block
  @Operation pprint(out,indent)
    format(out,"begin");
    @For s in statements do
      format(out,"~%~V",Seq{indent + 2});
      s.pprint(out,indent + 2)
    end;
    format(out,"~%~Vend",Seq{indent})
  end
\end{lstlisting}
An {\tt if} statement is pretty-printed as follows:
\begin{lstlisting}
context If
  @Operation pprint(out,indent)
    format(out,"if ");
    test.pprint(out);
    format(out,"~%~Vthen~%~V",Seq{indent,indent + 2});
    thenPart.pprint(out,indent+2);
    format(out,"~%~Velse~%~V",Seq{indent,indent + 2});
    elsePart.pprint(out,indent+2);
    format(out,"~%~Vend",Seq{indent})
  end
\end{lstlisting}
A type declaration is pretty-printed as follows, note the use of
{\tt ~\{} to iterate through the sequence of field names in the
{\tt format} control string:
\begin{lstlisting}
context TypeDeclaration
  @Operation pprint(out,indent)
    format(out,"type ~S is ~{,~;~S~} end",Seq{name,names})
  end
\end{lstlisting}
A value declaration:
\begin{lstlisting}
context ValueDeclaration
  @Operation pprint(out,indent)
    format(out,"value ~S is ",Seq{name});
    value.pprint(out);
    format(out," end")
  end
\end{lstlisting}
A {\tt while} statement:
\begin{lstlisting}
context While
  @Operation pprint(out,indent)
    format(out,"while ");
    test.pprint(out);
    format(out," do~%~V",Seq{indent+2});
    body.pprint(out,indent+2);
    format(out,"~%~Vend",Seq{indent})
  end
\end{lstlisting}\section{Conclusion}

This chapter has shown how XMF can be used to define the
operational semantics of languages. We have shown how to implement
an interpreter for a simple language and how to translate the
language to existing XMF languages. We have discussed a number of
different issues relating to language translation, in particular
how much work is performed statically and how much is left to
run-time.
