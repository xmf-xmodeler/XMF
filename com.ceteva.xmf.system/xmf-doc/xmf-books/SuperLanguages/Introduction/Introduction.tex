\chapter{Introduction}

The days of monolithic languages are over. The idea that one size
fits all is dead. Long live diversity! 

Applications should not, and indeed cannot, be written in a single
language these days. We have lots of technologies that are specific
to different aspects of complete application. We have SQL for databases,
we have JSP for web presentation, we have XML everywhere, we have
Java and J2EE, we have frameworks and libraries. So everything is
great - just map you application to an appropriate technology and
away you go! \normalsize

\begin{quotation}
{\em But wait a minute, perhaps this diversity is not such a good
thing. Representing our application in these technologies can be quite
hard. I understand how the application breaks down into aspects, but
those aspects don't seem to map down onto different aspects of the
technology very well. OK, so, I'll put a bit of each application feature
into the appropriate technologies. It takes a bit of translation,
but our guys are experts in the technologies so there should be no
problem. Great - the system works, everyone is happy.

Uh, oh - the requirements have changed. No matter - we'll just
talk it through with the customer and modify the implementation accordingly.
It will take a bit of time to trace the requirements through to the
implementation on all those different bits of technology, er how did
we implement that customer transaction again? Never mind, we can put
that other project on hold while the original developers are brought
back to reverse engineer the code.

Yikes! the original developers have left. Just great, now I
have to pick through all those technologies and try to do it myself.
Crikey - I don't really understand this stuff, it talks about accrual
periods and debentures. I know, I'll show it to the customer, surely
they can extract some meaning from this mess. Aargh - the customer
won't touch it and the new development team just wants to rewrite
it!}
\end{quotation}

\normalsize


\section{The Representation Chasm}

We are drowning in implementation technology. Software language engineers
have been very busy over the last 10 years inventing languages that
control the new frontiers of technology. In particular there are a
huge number of languages that provide control over distributed applications,
databases and web-content. Basic technology issues are very well provided
for. This can be seen in the rise of open-source technologies and
the standardisation of architectures for information processing applications. 

And yet there are still problems. Little attention has been paid to
closing the gap between the languages and the applications that they
implement: the \textbf{representation chasm}. The chasm is bridged
by highly skilled developers who take designs and produce the application.
But what if your developers are not as highly skilled as you would
like? This leads to errors and delays arising from misunderstandings.
Iterative development is compromised by the representation chasm since
various project stakeholders become disenfranchised depending on which
side of the chasm they live. Maintenance become difficult: the customers,
architects and designers shout new requirements from one side of the
chasm only for the development team to mis-hear them on the other.

How do we bridge the representation chasm? The answer is in moving
towards a new breed of language technology: \textbf{super-languages}.
A super-language provides a number of technical features that are
specifically designed to close the gap between the concepts in the
application domain and the technologies that are used to implement
the application. Super-languages can be used to empower development
teams by increasing the sophistication of the tools they use to implement
the application. 

Super-languages can be used to de-skill various aspects of the implementation
by offering restricted languages suitable for the application. Super-languages
can be made accessible to project stakeholders that would usually
be unable to engage with implementation technologies. Super-languages
can radically increase the efficiency of the development team and
can radically increase the quality of the delivered products, derisk
the development process, and increase the ability of the supplier
to maintain the products.


\section{Super-Languages}

What is a super-language and how does it differ from the current breed
of implementation technologies? An idealised super-language provides
control over all aspects of representation and execution. A super-language
can be extended with new features that make it easy to represent concepts
from the application that a customer would understand. These new concepts
can be weaved into the existing features of a super-language or may
be kept separate (as a different aspect for example). Existing execution
mechanisms in the super-language can be changed to reflect the needs
of each new application. Each new feature that is added to a super-language
has a description of how it should execute and how it integrates with
other features (where that is appropriate). 

A super-language must provide super-features that address: \emph{usability;
expressiveness; extensibility}. These are described in more detail
in the following sections.


\subsection{Usability}

A super-language must be fit for purpose. It should support the developer
in capturing features of the application and place no unnecessary
barriers in the way of development. The following features contribute
to the usability of a super-language:

\begin{description}
\item [{interactive}] A super-language cannot be a batch execution language.
Some form of top-level command interpreter is essential. Developers
want to interact with their programs and to deny them this pleasure
is cruel. Interactivity aids development and debugging, increases
developer productivity, and facilitates a deep understanding of execution
mechanisms. If a super-language is used for language engineering then
the developer will be implementing execution mechanisms which will
require a good understanding of how languages tick.
\item [{dynamic}] A super-language must be dynamic. It must be possible
to load new definitions into the language at any stage during execution.
Why wait to recompile and restart an application when it is not necessary?
A super-language may be used as the glue that binds an application
made up of many different modules written in a variety of languages.
In order to be glue, the super-language must provide mechanisms for
dynamically loading and processing the modules. This is closely related
to the idea of aspects.
\item [{reflection}] A super-language must be able to reflect on its own
structure and behaviour. It should be possible for the language to
query its own state and to represent and manipulate its own syntax.
Many of the domain specific languages that are represented by super-languages
will ultimately be represented in a basic core language (an alternative
is to be translated to some external language). Syntax translations
rely on having a complete representation of the super-language syntax
available to the language engineer.
\item [{interfaces}] Super-languages must not be closed worlds. It is unrealistic
to expect mainstream programming languages to become super-languages
overnight. In addition there are practical reasons why certain aspects
of a system may need to be implemented using conventional technology.
Therefore, until standard languages become super-languages, a super-language
should support interfaces to conventional technology. Super-languages
should support provision for running as embedded applications.
\end{description}

\subsection{Expressiveness}

Where possible, a super-language should abstract away from implementation
details in programs. It should be possible to quickly and efficiently
use a super-language for prototyping an application in addition to
product development. Key features are:

\begin{description}
\item [{high-level}] There is no excuse for a language which does not provide
features that make representation and computation easy to achieve.
These include things like heterogeneous lists; closures; pattern matching;
symbolic data; rich type systems etc. High-level features make a language
a pleasure to use and contribute to the readability of the code. Closures
are particularly important since they can be used to engineer applications
that would be complex and difficult to understand without them (particularly
search related applications).
\item [{dynamic(duck)typing}] Languages that are limited to static type
systems are too restrictive to be super-languages. It is difficult
to achieve interaction and meta-features with exclusively statically
typed languages. Trust the developer - they really do know what they
are doing! The type-system of a super-language should be extensible
just like all other aspects of the language. A super-language may
optionally impose static typing (static typing is a good thing), but
should not mandate it.
\item [{garbage-collection}] The days of direct pointer manipulation and
user defined memory management are surely behind us. All but the very
specialized real-time or hardware related applications cannot justify
the exposure of raw machine-level details in this way. 
\end{description}

\subsection{Extensibility}

A super-language must be extensible. In principle, all aspects of
the language from the syntax to the execution engine should be available
to the developer. Extensibility features must be fullsome and integrated
with the rest ofthe language. It \textbf{does not} count if you provide
a couple of system calls that changes the mode of the execution engine.
It \textbf{does} count if you can write a complete new execution algorithm
and plug it in. It \textbf{does not} count if you write a pre-processor
for your language. It \textbf{does} count if you can write a source
code translator and integrate it into the language compiler using
all of the context that the compiler manipulates. Key features are:

\begin{description}
\item [{aspects}] A super-language must support the idea of aspects whereby
new definitions can be added to existing language components. Ideally
the super-language will be written in itself and must be extensible
so that it can be tailored through extension to suit different application
domains. Groups of definitions added to existing components are grouped
into logically related packages or \emph{aspects}. It should be possible
to manipulate aspects as a unit, minimally in terms of addition, deletion
and debugging.
\item [{reflexive}] A super-language should be reflexive. This is related
to reflection, but differs in that the language \emph{can affect}
its own behaviour. Debuggers for languages rely on the ability to
be reflexive since they have to change the trace state of the operations
that they are tracing. An ideal super-language will be meta-circular
since this is the best way of ensuring (given the characteristic features
listed here) that the language will be sufficiently expressive.
\item [{meta-protocols}] Execution engines for languages are like software
frameworks. To plug into the framework you supply program code written
in a language that the framework understands. The framework executes
and processes the program code. The key steps in execution forms an
execution procotol for the framework. The execution protocol for a
super-language must be clearly defined. A \emph{meta-protocol }is
a collection of features that can be used to extend or redefine the
execution protocol. A super-language should offer a meta-protocol
that allows new languages to hook into the existing execution framework,
but which allows them to change it. A typical excample is the protocol
for an object-oriented language that includes: send; setSlot; getSlot
etc. A \emph{meta-object-protocol} (or MOP) allows you to replace
the definition of these features for specific types of object. 
\item [{lightweight-extensible-syntax}] The syntax of a super-language
must be extensible and ultimately replaceable. In order to be a power-tool
for developers, the super-language must allow new language features
to be added and weaved into the existing language. New language features
must be able to take full advantage of the execution framework for
the super-language including accesing context information such as
local variables etc. Super-languages should support the definition
of new language features that support standard programming idioms
and also support the publication of such definitions for others to
reuse.
\item [{heavyweight-extensible-syntax}] A super-language should provide
features that support the definition of complete external languages.
Such language should support a development method that allows different
aspects of a system to be represented by DSLs that are accessible
to external stakeholders and which may also deskill the development
process. These languages are not necessarily weaved into the main
super-language, instead they are used externally to the main language
and loaded on demand.
\end{description}

\section{The Pros and Cons of Super-Languages}

The benefits of super-languages have been outlined above. They empower
the developer to be more productive by raising the abstraction level
of the programming language and inventing standard representations
for common idioms. This technique has been used in many programming
languages from the C-preprocessor through to Lisp. New language constructs
can be standardized and shared.

Super-languages also provide an unprecedented degree of control over
the language engine via meta-features. Access to meta-features provides
a way of tailoring a language in a modular way without polluting programs.
This separation of concerns is very powerful and can be used to support
a mixed paradigm application on the same language platform without
compromising the readability or maintainability of the application.

If used to implement multiple heavyweight or external languages within
the same application, super-languages provide the basis for addressing
some of the significant concerns raised at the beginning of the chapter
relating to current implementation technology and the representation
chasm. This can lead to significant commercial benefits arising from
the ability to engage more stakeholders will all aspects of an application
and the ability to control and deskill aspects of the application
development.

Super-languages are not without their problems. Since they offer advanced
features they can be used inappropriately and can be challenging for
the novice programmer. The techniques used to implement super-languages
are not as mature as those for more conventional languages and therefore
efficiency and useability issues can be an issue. It is important
to put these issues in perspective. Super-languages represent a technology
goal for current generation programming languages to aim for. The
software industry must embrace super-languages and help to develop
the technologies and methods since they represent a very significant
step forward in software application development. Existing mainstream
languages should strive to adopt super-features and thereby move forward.


\section{Related Concepts}

The concept of super-languages is related to several other terms used
by the software industry. These are:

\begin{description}
\item [{LOP}] Language Oriented Programming is a term that is used to mean
the process of defining new programming concepts in order to address
a given application. A super-language supports LOP although not all
systems that support LOP need be super-languages. It is not necessary
to be a programming language in the first place to support LOP - you
just need to be a system that allows the definition of programming
languages. It is certainly not necessary to be meta-circular to provide
LOP.
\item [{DSL}] Domain Specific Languages is a term that is used to mean
a language that has been specifically designed to represent concepts
from an application domain. A super-language supports DSLs, however
the term DSL can mean languages that do not involve programming in
any way. The term includes graphical languages, such as profiles in
UML, that are used to express designs in terms of concepts and relationships.
\item [{LDD}] Language Driven Development is a method that can be used
to produce software applications. LDD involves the use of DSLs at
various points in the development life-cycle. If LDD is used as part
of the implementation phase of the project then the use of super-languages
is almost essential.
\item [{MDA}] Model Driven Architecture is a term used to mean the generation
of program code from UML models, possibly involving the use of DSLs
or UML profiles. Super-languages are not specifically required for
this process (although they may help in the transformation process
- see QVT).
\item [{Software-Factories}] Software factories is a term used to mean
the application of standard industry practices to software development.
More specifically it has been used to refer to an approach supported
by Visual Studio to develop languages that support the construction
of software components. This is similar in many ways to MDA, but is
not limited to using UML.
\item [{Language-Workbenches}] The term language workbench was proposed
by Martin Fowler in http://martinfowler.com/articles/languageWorkbench.html
to mean the IDE support for language oriented programming. Examples
of language workbenches include Intentional Software, the Meta programming
System from JetBrains and the Visual Studio provision for Software
Factories. A super-language is \emph{not} an IDE. Programming and
development processes that are based on super-languages do not require
IDE support. However, given that super-languages are by nature more
sophisticated than conventional languages, IDE support and language
workbenches are an important tool.
\end{description}

\section{Examples of Super-Languages}

The perfect super-language does not exist. Such a language would be
arbitrarily extensible, as efficient and usable as current languages,
have IDE support for all aspects of the language and would integrate
with anything. The perfect super-language is an ideal that should
motivate current language developers to strive for better features
that can address the representation chasm.

Languages with super-features do exist and have done so for quite
a while. Smalltalk and Lisp offer a significant number of super-features
although both frighten the horses in terms of scary syntax. 

C\# and Java offer a degree of reflection, but offer no support for
language extension and high-level programming features. It would be
very interesting to chart out a course of language evolution for these
languages which incrementally adds super-features.

Languages such as Ruby and Python offer a range of super-features,
including dynamic typing and meta-features, but are generally fixed
in their syntax. It is interesting to note that this style of language
(often referred to as \emph{scripting languages}) are generally popular
with technology lead programmers. This is a good indicator that languages
with super-features are likely to find their way into the mainstream.


\section{The Super-language XMF}

This book is about a language system called XMF. XMF has been designed
as a super-language and offers a full range of super-features. XMF
is not the perfect super-language, but it claims to be the language
that exhibits more super-features than any other language currently
available. XMF has been in development since around 2000 and has gone
through a number of development cycles. Its roots are in the Object
Management Group initiative to define the UML 2.0 standard when an
early version of XMF was developed in order to process fragments of
UML language definition. In case anyone is wondering, this explains
the terms OCL and XOCL that occur in XMF and also explains some of
the syntax used in XMF. The relationship between XMF and UML/OCL was
dropped many moons ago, leaving some of the features that contribute
to being a super-language and dropping all others.

Since 2003, XMF has been actively developed as a commercial product
and used by real paying customers as part of a larger product called
XMF-Mosaic. The Mosaic component was a large library of XMF code (over
100K lines of XOCL) that implemented diagram tools and property editors
generated from XMF meta-descriptions. Mosaic allowed XMF to run inside
Eclipse. At the time, Mosaic was ahead of the game, but since we started
development on Mosaic, systems like EMF/GMF and Software Factories
have caught up and overtaken Mosaic in terms of development effort.
In any case, Mosaic was a sophisticated working example of a fairly
large XMF application and shook most of the bugs out of the system.

The key super-strengths of XMF:

\begin{itemize}
\item it is written in itself and therefore it is highly self consistent.
Since extensibility is part of the language, this makes XMF completely
extensible.
\item it provides extensive support for working with syntax, both concrete
and abstract. XMF is a true LOP engine.
\item it provides very high-level programming features including pattern
matching, undoability, daemons, heterogeneous lists.
\item it implements a MOP and is truly meta-circular. The meta-language
is called XCore which defines all the data types and behaviour for
the rest of the system. There is no fudging of meta-classes as occurs
in some language systems.
\item it is interactive, dynamic and multi-threaded.
\item it can be embedded in Java both calling out to Java and being called
from Java.
\end{itemize}
The weaknesses of XMF as a super-language are:

\begin{itemize}
\item it is not as fast as a mainstream language. For many super-applications
this will not be a problem.
\item error handling is adequate, but no special support is provided for
new languages engineered with XMF.
\item there is no static typing (although this is planned for the next major
release).
\end{itemize}

\section{Guide to Book}

The aim of this book is to introduce you to the super-language XMF.
This will be done by defining the language, providing some examples of
applications that can be written directly in the XOCL language that
comes with XMF, and then by showing how you can use XMF for language
engineering.

The main focus of this book is on language engineering by example.
Although we introduce the XOCL language in general terms, we quickly focus
on the features that make XMF good for writing new languages and then
give lots of examples.

There are many super-features of XMF that we do not cover. These features include:
Java interface; daemons; undo. We don't spend very much time on the
MOP and the meta-interfaces of XMF. 

XMF is open-source technology. You can get a copy of XMF from {\tt www.ceteva.com}.
Since XMF is written in itself, it is one of the best examples of 
how to write XMF programs. In addition to the material in this book
we would encourage you to check out the XMF source code for examples
of how to write languages and how to program in XMF. 

This book is divided into the following parts:

\begin{description}
\item [{XOCL}] The XMF programming system consists of a
language called XOCL that runs on the XMF VM. In order to program
in the XMF system you will need to learn XOCL. This part includes
a tutorial on XOCL and the main data that runs on the XMF VM.
\item [{Programming}] This part of the book describes a number of
applications written in XOCL. None of the applications involve language
engineering, just using XOCL to produce a program that runs on XMF.
\item [{Language-Engineering}] Super-languages allow new language constructs
and even complete languages to be defined. This part of the book provides
a number of examples of how you can use XMF to engineer languages.
\end{description}

\section{References:}

\begin{itemize}
\item http://www.cse.dmu.ac.uk/\textasciitilde{}mward/martin/papers/middle-out-t.pdf
\item http://www.onboard.jetbrains.com/articles/04/10/lop/
\item http://martinfowler.com/articles/languageWorkbench.html
\item http://www.infoq.com/presentations/domain-specific-languages
\item http://www.martinfowler.com/
\end{itemize}