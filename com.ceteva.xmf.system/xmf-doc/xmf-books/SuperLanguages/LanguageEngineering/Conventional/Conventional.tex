\chapter{Language Emulation}

Super-languages provide support for syntax extension and for weaving new language
constructs into the base super-language. In addition they must allow new 
external languages to be defined so that applications can be constructed
by combining components that are written in mixtures of languages.

In addition to inventing new language constructs that capture programming
idioms and that provide domain specific abstractions, a super-language
should allow constructs from existing conventional languages to be used.
Ultimately, a super-language should provide support for complete emulation
of an existing conventional language or some sub-set of it.

A super-language must be able to assimilate conventional languages because
most projects do not start with a clean sheet. Existing code must be weaved
into a new application and a super-language should be able to accommodate
legacy code whilst providing scope for DSLs and new language features.

This chapter describes how XMF can be used to support features from existing
conventional languages.

\section{MicroJava}

XMF is shipped with an implementation of Java. The definition is found in 
the package Languages::MicroJava. MicroJava is a sub-set of the Java language
that supports classes, fields and methods. The MicroJava grammar supports
most Java features and is a useful starting point for designing Java-like
languages.

The MicroJava language does not support direct execution of programs. There
is no MicroJava compiler or interpreter. MicroJava is intended as a starting 
point for experimentation with Java or as a mechanism for exporting Java
code from XMF. MicroJava supports:
\begin{itemize}
\item A grammar that translates MicroJava concrete syntax into an instance
of the MicroJava model.
\item A pretty-printer that exports a textual representation of MicroJava
source code (a Java export mechanism).
\item A translation from MicroJava abstract syntax to XOCL abstract syntax.
\end{itemize}
Therefore, to execute MicroJava you can: pretty-print it and use a Java
compiler on the result; translate it to XOCL and execute that; write your
own MicroJava compiler or interpreter.

The grammar for MicroJava is defined by the class Java in the Languages::MicroJava
package. The following shows an example MicroJava program defined in a standard
XMF source file:
\begin{lstlisting}
parserImport XOCL;
parserImport Languages::MicroJava;
 
context Root
 
@Java
 
  public class Printer {
 
    private int copies;
    private String string;
 
    public Test(int copies,String string) {
    }
 
    private boolean printCount() {
      while(copies > 0) {
        this.write(string);
	    this.copies = this.copies - 1;
      }
      return true;
    }
 
    // Embedded XOCL
 
    with JOCL {
      @Operation write(string:String)
        format(stdout,"~S~%",Seq{string})
      end 
    }
  }
 
end
\end{lstlisting}
The @Java construct will cause the MicroJava program text to be parsed,
an instance of the MicroJava abstract syntax model to be created and then 
the translator to XOCL to be performed. The result of the parse is XOCL
abstract syntax which is then compiled or interpreted by the XOCL
execution system as normal.

You could use MicroJava to experiment with interesting language
extensions to Java. Each new construct can be added to the MicroJava
abstract syntax model. If you add a toXOCL operation to the new
abstract syntax class then the new MicroJava construct can be executed
via a translation to XOCL. See the source code for MicroJava for more
details about how the toXOCL operation works for each abstract syntax
class.

Notice in the MicroJava program shown above the use of a non-Java 
construct: {\tt with}. The {\tt with} keyword is followed by the
name of a class (that must be in-scope) that defines (via a grammar) 
a language. The source code between {\tt \{} and {\tt \}} in the {\tt with}
is written in the language.

The {\tt with} construct allows arbitrary languages to be nested
in MicroJava. In the example above, XOCL has been nested in MicroJava using
the following class definition for JOCL:
\begin{lstlisting}
context Root
  @Class JOCL
    @Grammar extends OCL::OCL.grammar
      // A grammar invoked using @X must be defined
      // by a class named X and have a clause named X.
      // The JOCL clause just proxies the SimpleExp
      // clause defined by OCL...
      JOCL ::= SimpleExp.
    end
  end
\end{lstlisting}

\section{Lisp}

Lisp is an interesting language because its syntax is very unconventional.
Lisp has a very small meta-syntax (just lists and atoms). The language is
built around this syntax in two ways:
\begin{itemize}
\item By adding builtin operators. The more orthogonal builtin operators are
added to a Lisp engine, the more applications can be built. 
\item By adding special forms. A special form is a Lisp list that
is recognized by the Lisp interpreter or compiler and treated specially.
\end{itemize}
This section provides a complete mini-Lisp system including a package of
builtin operators that you can extend. Execution of the Lisp system is
performed by a translation to XOCL. The translator is called from the grammar
rules - so there is no abstract syntax representation of Lisp in this example
(but you could add one in if you wanted to manipulate the Lisp expressions).
The translator, shows how Lisp special forms are handled.

Here is an example Lisp program:
\begin{lstlisting}
parserImport XOCL;

import LispOps;

context Root
  @Operation testFact(n:Integer):Integer
    // Define a factorial function and supply the 
    // integer n...
    @Lisp
      (let ((fact (Y (lambda(fact)
                       (lambda(n)
                         (if (eql n 0)
                             1
                             (mult n (fact (sub n 1)))))))))
           (fact n))
    end
  end
\end{lstlisting}
The builtin operators are defined in the package LispOps. A Lisp
program is executed in the current scope so that Lisp variables 
are the same as XOCL variables.

Note that recursive definitions are handled using a Y operator
that is defined in LispOps. It is beyond the scope of this book
to go into the detailed semantics of Lisp and Y - you'll just have
to trust us that Y is the fixed point operator for which:
\begin{lstlisting}
((Y f) x) = x
\end{lstlisting}
The grammar for Lisp parses the text and construct a sequence
containing the Lisp program. The program is then passed to an
operator eval that is defined in the package LispAux which
translates the sequence to XOCL code. The grammar is defined
below:
\begin{lstlisting}
parserImport XOCL;
parserImport Parser::BNF;
 
import OCL;
import LispAux;
 
context Root 
  @Class Lisp 
    @Grammar  
      Lisp ::=  e = Exp  'end' { eval(e) }.
        
      Exp ::= 
        Atom      // A basic syntax element.
      | List      // A sequence of expressions.
      | Const     // A (back)quoted element.
      | Comma.    // An unquoted element.
      
      Atom ::= 
        Int       // Integers.
      | Str       // Strings.
      | Float     // Floats
      | n = Name  // Names are turned into variables for syntax.
        { Var(n) }.
      
      List ::= '(' es = Exp* ListTail^(es).
        
      ListTail(es) ::=       
        // A list tail may be a '.' followed by an element then ')'
        // or may just be a ')'...        
        '.' e = Exp ')' 
        { es->butLast + Seq{es->last | e} }
        
      | ')' { es }.
      
      Const ::=        
        // A constant is a quoted expression. Use backquote
        // which allows comma to be used to unquote within
        // the scope of a constant...       
        '`' e = Exp 
        { Seq{Var("BackQuote"),e} }.
        
      Comma ::=        
        // A comma can be used within the scope of a
        // backquote to unquote the next expression...       
        ',' e = Exp 
        { Seq{Var("Comma"),e} }.        
    end
  end
\end{lstlisting}
The auxiliary definition {\tt eval} and {\tt quote} perform opposite
translation duties. The {\tt eval} operation translates Lisp source
code that is to be evaluated into XOCL code. The {\tt quote} operation
translates Lisp source code that is protected by a quote (actually
a backquote character here which allows comma to be used). The 
function of these two operations is given below:
\begin{lstlisting}
parserImport XOCL;
parserImport Parser::BNF;

import OCL;
  
context Root
  @Package LispAux  
    // Operations that are used by the Lisp language when
    // translating the parsed syntax into XOCL...    
    @Operation eval(e):Performable    
      // This operation takes a single value produced by the Lisp
      // parser and produces an XOCL performable element...      
      @TypeCase(e)      
        // Atoms all produce the XOCL equivalent syntax...       
        Integer do
          e.lift()
        end        
        Float do
          e.lift()
        end        
        Var do        
          // Variables are evaluated in the surrounding lexical
          // environment...         
          e
        end        
        String do
          e.lift()
        end        
        Seq(Element) do        
          // Sequences may start with special names which are
          // evaluated specially...         
          @Case e of          
            Seq{Var("BackQuote"),e} do           
              // Suppress the evaluation of the expression e in the
              // XOCL code that is produced...             
              quote(e)
            end           
            Seq{Var("Comma"),e} do          
              // Only legal within the scope of a backquote and 
              // therefore can only be processed by the quote
              // operation defined below...              
              self.error("Comma found without backquote")
            end            
            Seq{Var("lambda"),args,body} do             
              // Create a lexical closure....             
              lambda(args,eval(body))
            end           
            Seq{Var("if"),test,e1,e2} do           
              // Choose which of the two expressions e1 and e2
              // to evaluate based on the result of the test...             
              [| if <eval(test)> then <eval(e1)> else <eval(e2)> end |]
            end           
            Seq{Var("let"),bindings,body} do           
              // A let-expression introduces identifier bindings into
              // the current lexical environment. This can be viewed
              // as simple sugar for the equivalent lambda-application...              
              let formals = bindings->collect(b | b->head);
                  actuals = bindings->collect(b | b->at(1))
              in eval(Seq{Seq{Var("lambda"),formals,body} | actuals})
              end
            end           
            Seq{op | args} do           
              // Evaluate the argument expressions...             
              [| let A = <SetExp("Seq",args->collect(arg | eval(arg)))>
                 in 
                    // Then invoke the operation...                   
                    <eval(op)>.invoke(self,A)
                 end
              |]
            end           
            Seq{} do             
              // nil is self evaluating...             
              [| Seq{} |]
            end
          end
        end
      end
    end
    
    @Operation quote(e):Performable   
      // This operation takes a singlelisp expression produced by the
      // grammar and returns an XOCL expression that protects the
      // evaluation (similar to lifting the value represented by the
      // expression)...      
      @Case e of     
        Seq{Var("Comma"),e} do       
          // Comma undoes the effect of backquote...          
          eval(e)
        end        
        Seq{h | t} do        
          // Create a pair...          
          [| Seq{ <quote(h)> | <quote(t)> } |]
        end        
        Seq{} do         
          // The empty sequence (nil in Lisp)...         
          [| Seq{} |] 
        end        
        Var(n) do        
          // Protected names are just strings (symbols in Lisp)...          
          n.lift()
        end        
        else         
          // Otherwise just return an XOCL expression
          // that recreates the atomic value...          
          e.lift()
      end
    end
    
    @Operation lambda(A,body:Performable):Operation     
      // Returns an XOCL operation expression that
      // expects the arguments represented vy the sequence
      // A...     
      Operation("lambda",args(A),NamedType(),body)
    end
    
    @Operation args(A)     
      // Turn a sequence of OCL::Var into a sequence of
      // OCL::Varp (because operation expressions in XOCL use
      // Varp as the class of simple named arguments)...      
      @Case A of      
        Seq{} do
          A
        end        
        Seq{Var(n) | A} do
          Seq{Varp(n) | args(A)} 
        end       
      end
    end
  end
\end{lstlisting}
You can see from the definition of {\tt eval} that the translation must
handle a number of special forms. These are all sequences whose head is
a variable with a special name. These are:
\begin{description}
\item{\tt lambda} A function. This must create a closure and can be
directly translated to an XOCL operation.
\item{\tt Quote} A protected sequence. The rest of the sequence is
handed to {\tt quote} which translates the elements to the equivalent
XOCl constant builting expressions. Note that {\tt quote} must detect
the occurrence of {\tt comma} which un-quotes the a Lisp expression.
Unquoting is handled by calling {\tt eval} again.
\item{\tt if} An is-expression just translates directly onto an XOCL
if-expression.
\item{\tt let} A let-expression performs parallel binding. This can be
achieved by a translation to a Lisp expression that creates local
variables by applying a lambda-function.
\end{description}
The builtin operators of Lisp are all defined in the LispOps package
which is imported for the scope of any Lisp program. You can add any
operation definitions you like into this package. Here are some
examples including the infamous Y operator:
\begin{lstlisting}
parserImport XOCL;
parserImport Parser::BNF;
 
import OCL;
    
context Root 
  @Package LispOps  
    // A package of basic Lisp operations. Think of these
    // as the builtin operations and make sure you import
    // them before referencing them...  
    @Operation add():Integer    
      // Adding up integers...      
      args->iterate(arg n = 0 | n + arg)
    end
    
    @Operation mult(.args:Seq(Integer)):Integer    
      // Multiplying integers...      
      args->iterate(arg n = 1 | n * arg)
    end
    
    @Operation sub(x:Integer,y:Integer):Integer    
      // Subtracting...      
      x - y
    end
    
    @Operation list(.args):Seq(Element)    
      // Creating a list from some elements...     
      args
    end
    
    @Operation Y(f)    
      // The infamous Y combinator for finding a fixed point...      
      let op = @Operation(x) f(@Operation(.y) (x(x)).invoke(null,y) end) end
      in op(op)
      end
    end
    
    @Operation eql(x,y):Boolean   
      // Comparing two elements...      
      x = y
    end    
  end
\end{lstlisting}