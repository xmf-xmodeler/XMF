
\chapter{Grammars}

Computer based languages are used as communication media; either system
to system or human to system. In either case, a computer must be able
to recognise phrases expressed in the language. When phrases are presented
to a computer they must be interpreted in some way. Typically, phrases
may be interpreted directly, as they are recognized, or some internal
representation for the phrases may be synthesized (abstract syntax)
and subsequently passed on to an internal module for processing.

There are a variety of formats by which language may be conveyed to
a computer; a common mechanism is by text. A standard way of defining
how to recognize and process text-based languages is by supplying
a \emph{grammar}. A grammar is a collection of rules that define the
legal sequences of text strings in a language. In addition to recognising
language phrases, a grammar may include actions that are to be performed
as sub-phrases are processed.

In order to use grammars to process language phrases we must supply
the grammar to a program called a \emph{parser}. The medium used to
express the grammar is usually text. BNF (and the extended version
EBNF) is a text-based collection of grammar formation rules. BNF defines
how to express a grammar and what the grammar means in terms of processing
languages. 

This chapter describes the XMF version of EBNF, called XBNF. XBNF
integrates EBNF with XOCL (and any language defined by XBNF). XMF
provides an XBNF parser that is supplied with an XBNF grammar and
will then parse and synthesize XMF-defined languages. XMF is a super-language
because (amongst other features) it supports the definition of new
language constructs that can be weaved into XOCL or used stand-alone.

Grammars (and therefore languages) in XMF are associated with classes.
Typically an XBNF grammar defines how to recognize legal sequences
of characters in a language and to syntheisize an instance of the
associated class (and populate the instance with instances of the
class's attribute types).


\section{Parsing and Syntheisizing}

This section describes the basics of using XBNF to define simple grammars
that recognize languages and synthesize XMF data values. A grammar
is an instance of the XMF class Parser::BNF::Grammar. A grammar consists
of a collection of \emph{clauses} each of which is a rule that defines
a non-terminal of the grammar. A non-terminal is a name (by convention
a non-terminal name starts with an upper case letter).

A clause has the following form:
\begin{lstlisting}
NAME ::= RULE .
\end{lstlisting}
where the RULE part defines how to recognize a sequence of input characters.
To illustrate the essential elements of a grammar definition we will
build a simple caculator that recognises arithmetic expressions and
executes (synthesizes) the expressions (integers) as the parse proceeds.

The grammar is constructed incrementally, the clauses are defined
in the context of an @Grammar:
\begin{lstlisting}
@Grammar
  CLAUSES
end
\end{lstlisting}Typically, a grammar has a starting non-terminal; this is the clause
that is used as the starting point of a parse. There is nothing special
about the starting non-terminal in the definition. In the case of
the calculator the starting non-terminal is Calc:

\begin{lstlisting}
Calc ::= Mult '=' 'end'.
\end{lstlisting}The clause for Calc defines that to recognize this non-terminal a
parse must recognize an Mult (defined below) followed by a \emph{terminal}
'='. A terminal is a sequences of characters in quotes; a terminal
successfully matches an input when the input corresponds to exactly
the characters, possibly preceded by whitespace. Therefore, a Calc
is recognized when a Mult is recognized followed by a =.

A Mult is a multiplicative expression, possibly involving the operators
{*} and /. We use a standard method of defining operator precedence
in the grammar, where addition operators bind tighter than multiplicative
operators. This is achieved using two different non-terminals:

\begin{lstlisting}
Mult ::= n1 = Add ( 
    '*' n2 = Mult { n1 * n2 } 
  | '/' n2 = Mult { n1 / n2 } 
  | { n1 } 
). 
\end{lstlisting}The clause for Mult shows a number of typical grammar features. A
Mult is successfully recognized when a Add is recognized followed
by an optional {*} or / operator.

Each non-terminal synthesizes a value when it successfully recognizes
some input. When one non-terminal \emph{calls} another (for example
Add is called by Mult) the value synthesized by the called non-terminal
may be optionally named in the definition of the calling non-terminal.
Once named, the syntheized value may be referenced in the rest of
the rule. Naming occurs a number of times in the definition of Mult
where the names n1 and n2 are used to refer to syntheisized numbers.

A clause may include optional parts separated by |. In general, if
a clause contains an optional component of the form A | B where A
and B are arbitrary components, then a parse is successful when either
A or B is successful. 

The clause for Mult contains three optional components that occur
after the initial Add is successful (hence the need for parentheses
around the optional components to force the Add to occur). Once the
Add is successful then the parse is successful when one of the following
occurs:

\begin{itemize}
\item a {*} is encountered followed by a Mult.
\item a + is encountered followed by a Mult
\item no further input is processed (this option must occur last since it
subsumes the previous two options -- options are tried in turn).
\end{itemize}
Clauses may contain \emph{actions}. A action is used to synthesize
arbitrary values and can refer to values that have been named previously
in the clause. An action is an XOCL expression enclosed in \{ and
\}. Just like calls to non-terminals in the body of a clause, an action
returns a value that may be named. If the last component performed
by the successful execution of a clause is an action then the value
produced by the action is the value returned by the clause.

The clause for Mult contains a number of actions that are used to
synthesize values (in this case evaluate numeric expressions). The
clause for Add is the same as that for Mult except that the clause
recognizes and synthesises addition expressions:

\begin{lstlisting}
Add ::= n1 = Int ( 
    '+' n2 = Add { n1 + n2 }
  | '-' n2 = Add { n1 - n2 }
  | { n1 } 
). 
\end{lstlisting}The complete grammar definition is given below:

\begin{lstlisting}
@Grammar
  Calc ::= Mult '=' 'end'.
  Mult ::= n1 = Add (
      '*' n2 = Mult { n1 * n2 }
    | '/' n2 = Mult { n1 / n2 }
    | { n1 }
  ).
   Add ::= n1 = Int (
       '+' n2 = Add { n1 + n2 }
    |  '-' n2 = Add { n1 - n2 }
    | { n1 }
  ). 
end
\end{lstlisting}
\section{Sequences of Elements}

A grammar rule may want to specify that, at a given position in the
parse, a sequence of character strings can occur. For example, this
occurs when parsing XML elements, where a composite element can have
any number of children elements each of which is another element.

XBNF provides the postfix operators {*} and + that apply to a clause
component X and define that the parse may recognize sequences of occurrences
of X. In the case of {*}, there may be 0 or more occurrences of X
and in the case of + there may be 1 or more occurrences of X. If X
synthesizes a value each time it is used in a parse then the use of
{*} and + synthesizes sequences of values, each value in the sequence
is a value syntheisized by X.

The following gramar shows an example of the use of {*}. XML elements
are trees; each node in the tree has a label. The following grammar
recognizes XML (without attributes and text elements) trees and synthesizes
an XMF sequence representation for the XML tree. The grammar shows
the use of the builtin non-terminal Name that parses XMF names and
synthesizes the name as a string:

\begin{lstlisting}
@Grammar
  Element ::= 
      SingleElement 
    | CompositeElement. 
  SingleElement ::= 
    '<' tag = Name '/>' 
    { Seq{tag} }.
  CompositeElement ::= 
    '<' tag = Name '>' 
      children = Element* 
    '<' Name '/>' 
    { Seq{tag | children} }. 
end
\end{lstlisting}
\section{Specializing Grammars}

Grammars may be specialized by adding extra clauses or extending existing
clauses. A grammar has an \emph{extends} clause that is followed by
comma separated references to parent grammars. The newly defined grammar
is the merge of the parent grammars and the newly defined clauses.
Any clauses in the parents or in the body of the new definition that
have the same names are merged into a single clause using the | operator.

For example, suppose we want to extend the grammar for XML given in
the previous section with the ability to recognize text elements. A
text element is supplied as a string within string-quotes. XBNF provides
a built-in non-terminal Str for strings. The new grammar is defined
as follows:

\begin{lstlisting}
@Grammar extends XML 
  Element ::= Str. 
end
\end{lstlisting}This is equivalent to the following grammar definition:

\begin{lstlisting}
@Grammar
  Element ::= 
      SingleElement 
    | CompositeElement
    | Str. 
  SingleElement ::= 
    '<' tag = Name '/>' 
    { Seq{tag} }.
  CompositeElement ::= 
    '<' tag = Name '>' 
      children = Element* 
    '<' Name '/>' 
    { Seq{tag | children} }. 
end
\end{lstlisting}
\section{Synthesizing Syntax}

Super-languages must support extensibility through language definition.
When a grammar defines a new language we will often want to synthesize
syntax values that are then passed to a compiler or an interpreter.
An XMF language definition often consists of a grammar that synthesizes
XOCL. Suppose we want to modify the calculator language defined above
to produce the arithmetic expression rather than execute the expressions
directly. Assuming that the OCL package is imported:

\begin{lstlisting}
@Grammar
  Calc ::= Mult '=' 'end'.
  Mult ::= n1 = Add (
      '*' n2 = Mult { BinExp(n1,"*",n2) }
    | '/' n2 = Mult { BinExp(n1,"/",n2) }
    | { n1 }
  ).
  Add ::= n1 = IntExp (
      '+' n2 = Add { BinExp(n1,"+",n2) }
    | '-' n2 = Add { BinExp(n1,"-",n2) }
    | { n1 }
  ). 
  IntExp :: n = Int { IntExp(n) }.
end
\end{lstlisting}Alternatively we might choose to use quasi-quotes:

\begin{lstlisting}
@Grammar
  Calc ::= Mult '=' 'end'.
  Mult ::= n1 = Add (
      '*' n2 = Mult { [| <n1> * <n2> |] }
    | '/' n2 = Mult { [| <n1> / <n2> |] }
    | { n1 }
  ).
  Add ::= n1 = IntExp (
      '+' n2 = Add { [| <n1> + <n2> |] }
    | '-' n2 = Add { [| <n1> - <n2> |] }
    | { n1 }
  ). 
  IntExp ::= n = Int { n.lift() }.
end
\end{lstlisting}
\section{Simple Language Constructs}

Language driven development involves working with and developing high-level
languages that allow us to focus on the the \emph{what}, rather than
the \emph{how} of applications. Languages can be defined in terms
of abstract syntax models or concrete syntax models. Grammars support
the definition of concrete syntax models and are used extensively
throughout the rest of this book.

A grammar for a language construct synthesizes instances of an abstract
syntax model. So long as the model has an execution engine then the
synthesized data can be executed. The syntax model and engine may
be provided, such as XOCL, or may be one that has been defined specially
for the application; the latter case is referred to as a \emph{domain
specific language}. 

This section provides many examples of new language constructs that
are defined using grammars. All of the syntax constructs are provided
as part of the XMF system and implemented as extensions to the basic
XOCL language. In most cases the grammars synthesize instances of
the XOCL model and can therefore be viewed as \emph{syntactic sugar}.
All of the language constructs are general purpose and should not
really be viewed as domain specific; however the techniques can be
used when defining domain specific constructs.


\subsection{When}

When is a simple language construct that behaves as sugar for an if-expression:

\begin{lstlisting}
@When guard do
  action
end
\end{lstlisting}It emphasizes that there is a guarded action and there is no alternative
action. Although there is no semantic difference between a when and
an if, it can be useful when reading code to see what was in the mind
of the developer. The grammar is as follows:

\begin{lstlisting}
@Grammar extends OCL::OCL.grammar
  When ::= guard = Exp 'do' action = Exp {
    [| if <guard> 
       then <action> 
       else "GUARD FAILS" 
       end 
    |]
  }.
end

\end{lstlisting}
\subsection{Cond}

Sometimes, if-expressions can become deeply nested and difficult to
read. A cond-expression is equivalent to a nested if-then-else, but
promotes all of the if-then parts to the top level:

\begin{lstlisting}
@Cond
  test1 do exp1 end
  test2 do exp2 end
  ...
  testn do expn end
  else exp
end
\end{lstlisting}The implementation of this construct provides an example of using
syntactic sugar. A class that inherits from XOCL::Sugar must implement
an operation desugar that is used to transform the receiver into a
performable element. These classes are often useful when the transformation
is complex and is better done as a separate step after the abstract
syntax synthesis. The grammar for the cond-expression is as follows:

\begin{lstlisting}
@Grammar extends OCL::OCL.grammar
  Cond ::= clauses = Clause* otherwise = Otherwise 'end' {
    Cond(clauses,otherwise)
  }.
  Clause ::= guard = Exp 'do' body = Exp 'end' {
    CondClause(guard,body)
  }.
  Otherwise ::= 'else' Exp | 
    { [| self.error("No else clause.") |] }.
end
\end{lstlisting}Cond inherits from XOCL::Sugar. A Cond has attributes, clauses (of
type Seq(CondClause)) and otherwise (of type Performable) and has
a desugar operation as follows:

\begin{lstlisting}
@Operation desugar():Performable
  clauses->reverse->iterate(clause exp = otherwise |
    [| if <clause.guard()>
       then <clause.body()>
       else <exp>
       end
    |])
end
\end{lstlisting}
\subsection{Classes and Packages}

Classes and packages are name spaces: they both contain named elements.
Although name spaces can contain any type of named element, packages
and classes are examples of name spaces that know about particular
types of named element. Packages know about sub-packages and classes.
Classes know about attributes, operations and constraints. 

To allow extensibility, name space definitions contain sequences of
element definitions. An element is added to a name space through the
add operation. Packages and classes are examples of name spaces that
hijack the add operation in order to manage particular types of known
named elements. The rest of this section defined package and class
definition language features. 

A second feature of name spaces is mutual-reference. Definitions within
a name space may refer to each other and the references must be resolved.
For example, this occurs when attribute types refer to classes in
the same package and when classes inherit from each other in a package
definition. One way to achieve resolution is via two-passes. The named
elements are added to the name space but all references are left unresolved.
A second pass then replaces all references with direct references
to the appropriate named elements. This section shows how this two-pass
system can be implemented.

Firstly, consider a package definition:

\begin{lstlisting}
@Package P
  @Class A
    @Attribute b : B end
  end
  @Class B extends A
    @Attribute a : A end
  end
end
\end{lstlisting}This definition is equivalent to the following expression (ignoring
the translation of the classes):

\begin{lstlisting}
  Package("P")
    .add(@Class A ... end)
    .add(@Class B ... end)
\end{lstlisting}The grammar for package definition is as follows:

\begin{lstlisting}
@Grammar extends OCL::OCL.grammar
  Package ::= n = Name defs = Exp* 'end' {
    defs->iterate(def package = [| Package(<n.lift()>) |] |
      [| <package>.add(<def>) |])
  }.
end
\end{lstlisting}Both class and attribute definitions contain references to named elements
that may be defined at the same time. For example in the package definition
P above, class A refers to B and vice versa. References are delayed
until all the definitions are created. Since they have been delayed,
the references must be resolved. But references cannot be resolved
until the outermost definition has been completed. An outermost definition
is introduced by a 'context' clause, therefore this is the point at
which resolution can be initiated. All elements have an init operation
that is used to initialise elements; name spaces hijack the init operation
so that delayed references can be resolved.

For example, attributes are defined by the following grammar (simplified
so that types are just names):

\begin{lstlisting}
@Grammar
  Attribute::= n = Name ':' t = Name 'end' {
    [| Attribute(<n.lift()>).type := <t.lift()> |]
  }.
end
\end{lstlisting}
The init operation defined for attributes is:

\begin{lstlisting}
@Operation init()
  if type.isKindOf(String)
  then self.type := owner.resolveRef(type)
  end;
  super()
end
\end{lstlisting}
Classes are created as follows:

\begin{lstlisting}
@Grammar extends OCL::OCL.grammar
  Class ::= n = Name ps = Parents defs = Exp* 'end' {
    ps->iterate(p classDef = 
      defs->iterate(def classDef = [| Class(<n.lift>) |] 
      | [| <classDef>.add(<def>) |]) 
    | [| <classDef>.addParent(<p.lift>) |])
  }.
  Parents ::=
    'extends' n = Name ns = (',' Name)* { Seq{n|ns} }
  | { Seq{"Object"}.
end
\end{lstlisting}
The context rule is as follows:

\begin{lstlisting}
  Context ::= 'context' nameSpace = Exp def = Exp {
    [| let namedElement = <def>;
           nameSpace = <nameSpace>
       in nameSpace.add(namedElement);
          namedElement.init()
       end |]
  }.
\end{lstlisting}Attributes in XOCL, usually occor inside class definitions and have
the following syntax:

\begin{lstlisting}
@Attribute <NAME> : <TYPE> [ = <INITEXP> ] [ <MODIFIERS> ] end
\end{lstlisting}The type of an attribute is an expression that references a classifier.
Usually this is done by name (if the classifier is in scope) or by
supplying a name-space path to the classifier. If the type is a collection
(set or a sequence) then it has the form Set(T) or Seq(T) for some
type T. When the attribute is instantiated to produce a slot, the
slot must be initialised. The initial value for classes is null. Each
basic type (String, Integer, Boolean, Float) has a predefined default
value ({}``'', 0, false, 0.0). The default value for a collection
type is the empty collection of the appropriate type. A specific attribute
may override the default value by supplying an initialisation expression
after the type.

An attribute within a class definition may supply modifiers. The modifiers
cause the slot to have properties and create operations within the
class. Attribute modifiers have the following form:

\begin{lstlisting}
'(' <ATTMOD> (',' <ATTMOD>)* ')'
\end{lstlisting}where an attribute modifier may be one of: ?, !, +, -, \textasciicircum{}.
The meaning of these modifiers is given below:

\begin{itemize}
\item ? defines an accesor operation for the attribute. For an attribute
named n, an operation named n() is created that returns the value
of the slot in an instance of the class.
\item ! defines an updater operation for the attribute. For an attribute
named n, an operation named setN(value) is created that sets the value
of the slot named n to value in an instance of the class.
\item + defines an extender operation for the attribute providing that the
type is a collection. For an attribute named n, an operation named
addToN(value) is created that extends the collection in the slot of
an instance of the class. If the type of the attribute is a sequence
type then the new value is added to the end of the sequence.
\item - defines a reducer operation for the attribute providing that the
type is a collection. For an attribute named n, an operation named
deleteFromN(value) is created that extends the collection in the slot
of an instance of the class.
\item \textasciicircum{} makes the slot a container providing that the type
inherits from XCore::Contained. In this case values in the slot will
be contained objects with a slot named owner. Using the updater and
extender operations to update the slot defined by this attribute will
modify the owner slot when the value is added. It is often desirable
to maintain linkage between a contained value and its owner, this
attribute modifier manages this relationship automatically providing
that the slot is always modified through the updater and extender.
\end{itemize}
The grammar for attribute definitions is given below. The grammar
synthesizes an instance of the syntax class XOCL::Attribute that is
then expanded to produce an expression that creates a new attribute
in addition to being used by a class definition to add in attribute
modifier operations:

\begin{lstlisting}
@Grammar extends OCL::OCL.grammar
  Attribute ::= 
    name = AttName       // Name may be a string.
    meta = MetaClass     // Defaults to XCore::Attribute
    ':' type = AttType   // An expression.
    init = AttInit       // Optional init value.
    mods = AttMods       // Optional modifiers.
   'end' { Attribute(name,mult,type,init,mods,meta) }.
  AttInit ::= 
    // Init values may be dynamic or static. A
    // dynamic init value is evaluated each time
    // a slot is created. A static init value is
    // evaluated once when the attribute is
    // created and used for all slots.
    '=' Exp 
  | '=' 'static' e = Exp { [| static(<e>) |] } 
  | { null }.
  AttMods ::= 
    mods = { AttributeModifiers() } 
    [ '(' AttModifier^(mods) (',' AttModifier^(mods))* ')' ] 
    { mods }.
  AttModifier(mods) ::= 
    mod = AttMod { mods.defineModifier(mod) }.
  AttMod ::= 
    '?' { "?" } 
  | '!' { "!" } 
  | '+' { "+" } 
  | '-' { "-" } 
  | '^' { "^" }.
  AttType ::= n = AttTypeName AttTypeTail^(n).
  AttTypeName ::= 
    n = Name ns = ('::' Name)* { Path(Var(n),ns) }.
  AttTypeTail(n) ::= 
    '(' args = CommaSepExps ')' { Apply(n,args) } 
  | { n }.
  AttName ::= Name | Str.
  MetaClass ::= 'metaclass' Exp | { null }.
end 
\end{lstlisting}
\subsection{Operations}

An operation has a basic form of:

\begin{lstlisting}
@Operation [ <NAME> ] <ARGS> <BODY> end
\end{lstlisting}The operation name may be optional and defaults to anonymous. The
following extensions to the form of basic operation definition are
noted:

\begin{itemize}
\item Operations have properties (allowing them to behave like objects with
slots). The properties are optional and specified between the operation
named and its arguments.
\item Arguments have optional types, specified as : <EXP> following the
argument name.
\item Argument names may be specified as XOCL patterns. The pattern matches
the supplied argument value.
\item Arguments are positional except in the case of an optional multiargs
specification as the last argument position. A multiargs specification
follows the special designator '.'. This is a mechanism that allows
operations to take any number of arguments. Arguments are matched
positionally and it is an error to supply less than the number of
declared positional arguments to an operation. If more than the expected
positional arguments are supplied and if the operation has a multiargs
designator then all the remaining supplied argument values are supplied
as a sequence and matched to the multiargs pattern.
\end{itemize}
The grammar for operation definition is shown below. the grammar synthesizes
an instance of XOCL::Operation (via mkOpDef) that is then analysed
and expanded to an operation creation expression:

\begin{lstlisting}
@Grammar extends OCL::OCL.grammar     
  Operation ::=        
    name = OpName 
    properties = Properties        
    '(' args = OpArgs multi = OpMulti ')'       
    type = ReturnType       
    body = Exp+        
    'end'       
    { mkOpDef(name,properties,args,multi,type,body) } .
  OpName ::= 
    Name 
  | { Symbol("anonymous") }.     
  OpArgs ::= 
    arg = Pattern args = (',' Pattern)* { Seq{arg | args } } 
  | { Seq{} }.     
  OpMulti ::= 
    '.' multi = Pattern { Seq{multi} } 
  | { Seq{} }.     
  ReturnType ::= 
    ':' TypeExp 
  | { NamedType() }.     
  Properties ::= 
    '[' p = Property ps = (',' Property)* ']' { Seq{p|ps} } 
  | { Seq{} }.     
  Property ::= n = Name '=' e = Exp { Seq{n,e} }.
\end{lstlisting}
\subsection{Enum}

Enumerated types occur frequently in models. An enumerated type consists
of a collection of predefined values; each value is classified by
the enumerated type and is different from all other values of the
type and different from all values of any other type. Examples include
Colour with values Red, Green and Blue. Another example is Directions
with values North, East, West and South. Here is an example of an
enumerated type and its usage:

\begin{lstlisting}
context Root
  @Enum Colour(Red,Green,Amber) end

context Root
  @Enum Position(North,South,East,West) end

context Root
  @Class TrafficLight
    @Attribute colour : Colour (?,!) end
    @Attribute position : Position (?,!) end
    @Constructor(colour,position) ! end
  end

context Root
  @Class RoadCrossing
    @Attribute light1 : TrafficLight 
      = TrafficLight(Colour::Red,Position::North) 
    end
    @Attribute light2 : TrafficLight 
      = TrafficLight(Colour::Red,Position::South) 
    end
    @Attribute light3 : TrafficLight 
      = TrafficLight(Colour::Green,Position::East) 
    end
    @Attribute light4 : TrafficLight 
      = TrafficLight(Colour::Green,Position::West) 
    end
    @Operation cycle()
      // Cycle the traffic lights...
    end
  end
\end{lstlisting}Many systems implement enumerated types as strings or integers. This
is an approximation to the above specification for enumerated types
since, if Red = 0 and West = 0 then both Colour and Direction are
really Integer and Red can easily be confused with West. An enumerated
type should be a classifier, and values of the type should be classified
by it. 

This is a good example of the use of meta-classes to implement new
types. The following class definition declares a class Enum; its definition
shows a number of useful features when declaring new meta-classes.
Firstly the definition:

\begin{lstlisting}
context XCore
  @Class Enum extends Class
    @Attribute names : Seq(String) (?) end
    @Constructor(name,names) ! 
      @For name in names do
        self.addName(name)
      end
    end
    @Operation default()
      self.getElement(names->head)
    end
    @Operation defaultParents():Set(Classifier)
      Set{NamedElement}
    end
    @Operation add(n)
      if n.isKindOf(String)
      then self.addName(n)
      else super(n)
      end
    end
    @Operation addName(name:String)
      self.names := names->including(name);
      self.add(self(name))
    end
  end
\end{lstlisting}Enum is a sub-class of XCore::Class. this means that instances of
Enum are classes that have their own instances - the values of the
enumerated type. Therefore, Colour is an instance of Enum and Red
is an instance of Colour. The class Enum has an attribute names that
is used to hold the declared names of its instances. When a new instance
of Enum is created, it is supplied with a name (Colour) and a collection
of names (Red,Green,Amber). 

In addition to the names being recorded as the value of the slot names,
each name gives rise to an instance of the enumerated type. Each instance
of the enumerated type is a new value that has a name, therefore Colour
must inherit from NamedElement. When a meta-class is defined it may
redefine the operation inherited from Class called defaultParents.
This must return a set of classifiers that instances of the meta-class
will inherit from by default. In the case of Enum, any instance of
Enum is declared to inherit from NamedElement, therefore the following
creates two named elements that are instances of the class Colour:
Colour({}``Red'') and Colour({}``Green'').

The operation Enum::addName is called for each of the names declared
for the enumerated type. An instance of Enum is a class therefore
it is a name-space and can contain named elements. the operation addName
created an instance of the enumerated type self(name) and then adds
it to itself. Each name declared for the enumerated type gives rise
to a new instance which is named and is guaranteed to be different
(when compared with =) to any other value.

A grammar for Enum completes the definition of enumerated types. Notice
that, since an enumerated type is a class it can contain attributes
and operations in its body:
\begin{lstlisting}
@Grammar extends OCL::OCL.grammar
  Enum ::= n = Name ns = Names es = Exp* 'end' {
    exps->iterate(e x = [| Enum(<name.lift()>,<names.lift()>) |] | 
      [| <x>.add(<e>) |])
  }.
  Names ::= '(' n = EnumName ns = (',' EnumName)* ')' { 
    Seq{n | ns} 
  }.
  EnumName ::= 
    Name
  | Str.
end
\end{lstlisting}

\section{Looping Constructs}

A @For-loop has a syntax construct defined by XOCL::For. 
When you define a @For-loop, the system creates an instance 
of XOCL::For and then desugars it. The @For-loop is a good 
example of a fairly extensive syntax transformation since 
the construct has many different forms controlled by a number 
of keywords. This section defines the syntax of a @For-loop 
and provides the definition of one of the syntax transformations.

The class XOCL::For has the following attribites:
\begin{lstlisting}
// The controlled variables...
@Attribute names     : Seq(String)      end  
// The type of for loop...
@Attribute directive : String           end  
// A guard on the bound element (null = true)...
@Attribute guard     : Performable      end  
// The collections...
@Attribute colls     : Seq(Performable) end  
// The body performed in the scope of 'name'...
@Attribute body      : Performable      end  
// Returns a value...
@Attribute isExp     : Boolean          end  
\end{lstlisting}
The @For-expression has the following grammar:
\begin{lstlisting}
@Grammar extends OCL::OCL.grammar 
   For ::=  
     // A for-loop may bind a sequence of names...
     names = ForNames     
     // The directive defines how to interpret the
     // collections...    
     dir = ForDirective     
     // A collection for each of the names...    
     colls = ForColls     
     // An optional boolean expression that
     // must be true in order to process
     // the names...    
     guard = ('when' Exp | { null })    
     // We are either performing commands or
     // producing a sequence of values...     
     isExp = ForType     
     // The body - either a command or an 
     // expression..   
     body = Exp    
     'end'     
     { For(names,dir,colls,guard,body,isExp) }. 
       
   ForColls ::= exp = Exp exps = (',' Exp)* { Seq{exp | exps} }.
    
   ForDirective ::= 
     // The directive controls how the collections
     // are processed...  
     // A walker produces instances of named types...   
     'classifiedBy'  { "classifiedBy" }     
     // Select elements from the collections...       
   | 'in'            { "in" }           
     // Reverse the collections...                 
   | 'inReverse'     { "inReverse" }         
     // Get keys from a table...           
   | 'inTableKeys'   { "inTableKeys" }         
     // Get values from a table...         
   | 'inTableValues' { "inTableValues" }       
     // Call ->asSeq...        
   | 'inSeq'         { "inSeq" }.
         
   ForNames ::=      
     name = AName       
     names = (',' AName)*        
     { Seq{name | names} }.
         
   ForType ::=      
     'do' { false }        
   | 'produce' { true }.
    
 end 
\end{lstlisting}
Each of the different @For-directives defines a different desugar 
operation for the components of a @For-expression abstract syntax 
construct. The following shows how a basic @For-loop using the 'in' 
directive is desugared:
\begin{lstlisting}
context XOCL::For
  @Operation desugarInAction():Performable
     [| let <self.collBindings()>
        in let isFirst = true
           in @While not <self.emptyCollCheck()> do
                let <self.bindNames()>
                in <self.takeTails()>;
                   let isLast = <self.emptyCollCheck()>
                   in <self.protect(body)>;
                      isFirst := false
                   end
                end
              end
           end
        end
     |]
  end
\end{lstlisting}
The operations used by desugarInAction are defined below:
\begin{lstlisting}
context XOCL::For
  @Operation bindNames():Seq(ValueBinding)       
     // Bind each of the controlled variables to the head of the
     // corresponding collection.      
     0.to(names->size-1)->collect(i |
       ValueBinding(names->at(i),[| <Var("forColl" + i)> ->head |]))        
   end

context XOCL::For
  @Operation collBindings():Seq(ValueBinding)     
     // Returns a sequence of bindings that guarantee the collections
     // are all sequences...       
     0.to(colls->size-1)->collect(i |
       ValueBinding("forColl" + i,[| <colls->at(i)> ->asSeq |]))
   end

context XOCL::For
  @Operation emptyCollCheck():Performable     
    // Check that any of the collection variables is empty...     
    1.to(colls->size-1)->iterate(i test = [| <Var("forColl0")> ->isEmpty |] |
      [| <test> or <Var("forColl" + i)> ->isEmpty |])       
  end

context XOCL::For
  @Operation protect(exp:Performable)  
    // If the guard is defined then check it before performing
    // the expression. Otherwise perform the expression. If
    // we are an expression then add the result of the expression
    // to the for results; these will be returned at the end of
    // the evaluation.    
    if guard = null
    then 
      if isExp
      then [| forResults := Seq{<exp> | forResults} |]
      else exp 
      end
    else 
      if isExp
      then [| if <guard> 
              then forResults := Seq{<exp> | forResults} 
              end |]
      else [| if <guard> 
              then <exp> 
              end |]
      end
    end
  end

context XOCL::For
  @Operation takeTails():Performable
    1.to(colls->size-1)->iterate(i updates = [| <Var("forColl0")> := <Var("forColl0")> ->tail |] |
      [| <updates>; <Var("forColl" + i)> := <Var("forColl" + i)> ->tail |])
  end
\end{lstlisting}

\section{XBNF}

This section describes the syntax for defining XBNF grammars. Since
XMF is a super-language, it can be used to define languages. XBNF
is just a language and can therefore be defined in XMF. One of the
goals of a super-language is meta-circularity. If you look at the
source code of XMF you will find a definition of the XBNF grammar
written in the XBNF language as follows (start reading at the clause
called Grammar):

\begin{lstlisting}
@Grammar extends OCL::OCL.grammar
    XBNF_Action ::=   
      // An action is either an expression in { and } 
      // which synthesizes a value or is a predicate that
      // must be true for the parse to proceed...     
      '{' exp = Exp '}' 
      { PreAction(exp) } 
    | '?' boolExp = Exp
      { PrePredicate(boolExp) }.
    
    XBNF_Atom ::=    
      // An atom is the basic unit of a clause...     
      XBNF_Action    
    | XBNF_Literal   
    | XBNF_Call      
    | '(' XBNF_Disjunction ')'.
  
    XBNF_Binding ::=    
      // A clause binding performs a grammar action and
      // associates the value produced with a named 
      // local...     
      name = Name '=' atom = XBNF_Sequence { 
        And(atom,Bind(name)) 
      }.
    
    XBNF_Call ::=    
      // Call a clause. The arguments are optional...     
      name = Name args = XBNF_CallArgs { Call(name,args) }.
      
    XBNF_CallArgs ::=     
      // Arguments supplied to a clause are optional.
      // The args must be preceded by a ^ to distinguish
      // the args from a pair of calls with 0 args...   
      '^' '(' n = Name ns = (',' Name)* ')' { Seq{n|ns} } 
    | { Seq{} }.
  
    XBNF_Clause ::= 
      // A clause is a named rule for parsing...   
      name = Name args = XBNF_ClauseArgs '::=' 
      body = XBNF_Disjunction '.' 
      { Clause(name,args,body) }. 
  
    XBNF_ClauseArgs ::=    
      '(' n = Name ns = (',' Name)* ')' { Seq{n|ns} } 
    | { Seq{} }.
    
    XBNF_Conjunction ::=    
      // Conjunction is just a sequence of 1 or more
      // clause elements...    
      elements = XBNF_Element+ { 
       elements->tail->iterate(e conj = elements->head | 
         And(conj,e)) 
    }. 
  
    XBNF_Disjunction ::=    
      // A disjunction is a sequence of elements
      // separated by | ...   
      element = XBNF_Conjunction (
        '|' rest = XBNF_Disjunction { Or(element,rest) } 
      | { element }).
   
    XBNF_Element ::=     
      XBNF_Optional    
    | XBNF_Binding   
    | XBNF_Sequence.
  
    Grammar ::=     
      parents = XBNF_GrammarParents
      imports = XBNF_GrammarImports
      clauses = XBNF_Clause* 
      'end'
      { Grammar(parents,clauses->asSet,"",imports) }.
  
    XBNF_GrammarImports ::=    
      // The imports of a grammar affect the grammars that are
      // available via @...    
      'import' class = Exp classes = (',' Exp)* { Seq{class | classes} } 
    | { Seq{} }.
  
    XBNF_GrammarParents ::=     
      // A grammar may inherit from 0 or more parent grammars.
      // The parent clauses are added to the child...   
      'extends' parent = Exp parents = (',' Exp)* 
      { parents->asSet->including(parent) } 
    | { Set{} }.
 
    XBNF_Literal ::=     
      // The following literals are built-in non-terminals of a
      // grammar. The action uses getElement to reference the
      // classes (and therefore the constructors) because a grammar
      // cannot reference a variable with the same name as a terminal
      // in an action...     
      // Get the next character...
    
      'Char'       { (Parser::BNF.getElement("Char"))() }          
      // Get the next line...       
    | 'Line'       { (Parser::BNF.getElement("Line"))() }         
      // Get a string...      
    | 'Str'        { (Parser::BNF.getElement("Str"))() }          
      // Get a terminal (in ' and ')...      
    | 'Terminal'   { (Parser::BNF.getElement("Term"))() }         
      // Return the current token...     
    | 'Token'      { (Parser::BNF.getElement("Tok"))() }     
      // Get an integer...           
    | 'Int'        { (Parser::BNF.getElement("Int"))() }         
      // Get a float...      
    | 'Float'      { (Parser::BNF.getElement("Float"))() }       
      // Get a name...     
    | 'Name'       { (Parser::BNF.getElement("Name"))() }    
      // Expect end-of-file...         
    | 'EOF'        { (Parser::BNF.getElement("EOF"))() }         
      // Throw away all choice points created since starting
      // the current clause...      
    | '!'          { (Parser::BNF.getElement("Cut"))() }       
      // Dispatch to the grammar on the most recently
      // synthesized value which should be a sequence of
      // names the represent a path to a classifier with
      // respect to the currently imported name-spaces...         
    | '@'          { (Parser::BNF.getElement("At"))() }          
      // Add a name-space to the currently imported 
      // name-spaces...       
    | 'ImportAt'   { (Parser::BNF.getElement("ImportAt"))() }     
      // Get the current state of the parsing engine...      
    | 'pState'     { (Parser::BNF.getElement("PState"))() }      
      // Get the current line position...     
    | 'LinePos'    { (Parser::BNF.getElement("LinePos"))() }     
      // Define a new terminal in the form NewToken(NAME,ID)...     
    | XBNF_NewToken                                   
      // Get a terminal name...                      
    | terminal = Terminal { (Parser::BNF.getElement("Terminal"))(terminal) }.
    
    XBNF_NewToken ::=    
      // A new token is defined as a name and an integer id.
      // The tokenizer used to parse the grammar is responsible
      // for returning a token with the type set to the id...   
      'NewToken' '(' n = Name ',' t = Int ')' {
        (Parser::BNF.getElement("NewToken"))(n,t) 
    }.
 
    XBNF_Optional ::=     
      // An optional clause element is tried and ignored if it fails...      
      '[' opt = XBNF_Disjunction ']'
      { Opt(opt) }.
  
    XBNF_Path ::= name = Name names = ('::' Name)* { Seq{name | names} }.
    
    XBNF_TypeCheck ::=    
      // An element that checks the type of the synthesized value...    
      element = XBNF_Atom (':' type = XBNF_Path { And(element,TypeCheck(type)) } 
    | { element }).
  
    XBNF_Sequence ::=    
      // An element an be followed by a * or a + for 0
      // or more and 1 or more repetitions...    
      element = XBNF_TypeCheck ( 
        '*' { StarCons(element) } 
      | '+' { PlusCons(element) } 
      | { element }
      ).
    
end
\end{lstlisting}
